%!TEX root = MemoireZelliges.tex

\chapter{Spectrométrie d'absorption optique : spectres de référence 
des colorants ioniques}

D'après \cite{Lajarte_1979}, épaisseur des échantillons : \SI{6}{\mm}.

  \pgfplotsset{
    % every axis plot/.style={black, thick},
    minor grid style={ultra thin, densely dotted},
    legend style={
      draw=none, font=\scriptsize
    }
  }

\noindent\hspace*{-1cm}%
\begin{tikzpicture}
  \pgfplotsset{
    every axis plot/.style={smooth, tension=0.6},
    legend style={
      legend plot pos=none,
      nodes={text width=1.9cm, text depth=, inner sep=0, align=right}
    }
  }

  \begin{groupplot}[
    group style={
      group size=2 by 3,
      horizontal sep=4mm,
      vertical sep=5mm,
      x descriptions at=edge bottom,
      y descriptions at=edge left,
    },
    width=9.5cm, height=6cm,
    enlargelimits=false,
    clip=false,
    xlabel={Longueur d'onde (\si{\nm})},
    ylabel={Transmission (\si{\percent})},
    xmin=300, xmax=1200,
    minor x tick num=1,
    ymin=0, ymax=100,
    minor y tick num=1,
    grid=both,
    ticklabel style={font=\small},
    no markers,
  ]

  \nextgroupplot[legend pos=south east]
    \pgfplotstableread{../data/ref/raw/lajarte_Cr_transf.dat}{\dataCr}
    \addplot table [x=lambda, y=trans] {\dataCr} ;
    \addlegendentry{Verre au chrome (\SI{0.05}{\percent} \ch{Cr2O3})}

  \nextgroupplot[legend pos=south east]
    \pgfplotstableread{../data/ref/raw/lajarte_Mn_transf.dat}{\dataMn}
    \addplot table [x=lambda, y=trans] {\dataMn} ;
    \addlegendentry[text width=1.4cm]
                   {Verre au manganèse (\SI{1}{\percent} \ch{MnO})}

  \nextgroupplot
    \pgfplotstableread{../data/ref/raw/lajarte_Fe_transf.dat}{\dataFe}
    \addplot table [x=lambda, y=trans] {\dataFe} ;
    \addlegendentry{Verre au fer (\SI{0.80}{\percent} \ch{Fe2O3})}

  \nextgroupplot
    \pgfplotstableread{../data/ref/raw/lajarte_Ni_transf.dat}{\dataNi}
    \addplot table [x=lambda, y=trans] {\dataNi} ;
    \addlegendentry{Verre au nickel (\SI{0.20}{\percent} \ch{Ni2O3})}

  \nextgroupplot
    \pgfplotstableread{../data/ref/raw/lajarte_Cu_transf.dat}{\dataCu}
    \addplot table [x=lambda, y=trans] {\dataCu} ;
    \addlegendentry{Verre au cuivre (\SI{0.05}{\percent} \ch{CuO})}

  \nextgroupplot[legend pos=south east]
    \pgfplotstableread{../data/ref/raw/lajarte_Co_transf.dat}{\dataCo}
    \addplot table [x=lambda, y=trans] {\dataCo} ;
    \addlegendentry{Verre au cobalt (\SI{0.05}{\percent} \ch{CoO})}
  \end{groupplot}
\end{tikzpicture}

\newpage

D'après \cite{Stroud_1971}.
Absorption causée par le nickel (\ch{Ni}), le manganèse (\ch{Mn}), 
le cuivre (\ch{Cu}), le fer (\ch{Fe}) et le chrome (\ch{Cr}) 
dans des verres plombifères et calco-sodiques.

\bigskip

\pgfplotstableread{../data/ref/raw/stroud_Mn.dat}{\dataMn}
\pgfplotstableread{../data/ref/raw/stroud_Cu.dat}{\dataCu}
\pgfplotstableread{../data/ref/raw/stroud_Ni.dat}{\dataNi}

\noindent\hspace*{-1.5cm}%
\begin{tikzpicture}
  % \pgfplotsset{
  %   legend style={
  %     nodes={align=left},
  %   }
  % }
  \begin{groupplot}[
    group style={
      group size=3 by 1,
      horizontal sep=8mm,
      % vertical sep=5mm,
      x descriptions at=edge bottom,
      y descriptions at=edge left,
    },
    width=6cm, height=9cm,
    enlargelimits=false,
    % clip=false,
    unbounded coords=jump,
    xlabel={Longueur d'onde (\si{\nm})},
    ylabel={Absorption (\si{\per\cm})},
    ymin=0, ymax=1.5,
    minor y tick num=1,
    grid=both,
    ticklabel style={font=\small},
    no markers,
    legend cell align=left,
  ]

  \nextgroupplot[%
    % legend pos=south east
    xmin=390, xmax=700,
    minor x tick num=1,
  ]
    \addplot table [x=lambda_p, y=abs_p] {\dataNi} ;
    \addlegendentry{Verre plombifère}
    \addplot table [x=lambda_c, y=abs_c] {\dataNi} ;
    \addlegendentry{Verre calco-sodique}
    \node [fill=white] 
          at (axis description cs:0.78,0.75) {\bfseries \ch{Ni}} ;

  \nextgroupplot[%
    % legend pos=south east
    xmin=400, xmax=900,
    minor x tick num=1,
  ]
    \addplot table [x=lambda_p, y=abs_p] {\dataMn} ;
    \addlegendentry{Verre plombifère}
    \addplot table [x=lambda_c, y=abs_c] {\dataMn} ;
    \addlegendentry{Verre calco-sodique}
    \node [fill=white] 
          at (axis description cs:0.78,0.75) {\bfseries \ch{Mn}} ;

  \nextgroupplot[%
    % legend pos=south east
    xmin=400, xmax=1400,
    minor x tick num=1,
  ]
    \addplot table [x=lambda_p, y=abs_p] {\dataCu} ;
    \addlegendentry{Verre plombifère}
    \addplot table [x=lambda_c, y=abs_c] {\dataCu} ;
    \addlegendentry{Verre calco-sodique}
    \node [fill=white] 
          at (axis description cs:0.78,0.75) {\bfseries \ch{Cu}} ;
  \end{groupplot}
\end{tikzpicture}

\bigskip
\bigskip

\pgfplotstableread{../data/ref/raw/stroud_Fe.dat}{\dataFe}
\pgfplotstableread{../data/ref/raw/stroud_Cr.dat}{\dataCr}

\noindent\hspace*{-1.5cm}%
\begin{tikzpicture}

  \begin{groupplot}[
    group style={
      group size=2 by 1,
      horizontal sep=15mm,
      % vertical sep=5mm,
      x descriptions at=edge bottom,
      % y descriptions at=edge left,
      ylabels at=edge left,
    },
    width=6cm, height=9cm,
    enlargelimits=false,
    % clip=false,
    unbounded coords=jump,
    xlabel={Longueur d'onde (\si{\nm})},
    ylabel={$\alpha / \text{N}$ (\si{\cm\squared})},
    ymode=log,
    % ymin=0, ymax=1.5,
    % minor y tick num=1,
    grid=both,
    ticklabel style={font=\small},
    no markers,
    legend cell align=left,
  ]

  \nextgroupplot[%
    % legend pos=south east
    xmin=350, xmax=1300,
    minor x tick num=1,
    ymin=1e-19, ymax=2e-17,
  ]
    \addplot table [x=lambda_p, y=abs_p] {\dataFe} ;
    \addlegendentry{Verre plombifère}
    \addplot table [x=lambda_c, y=abs_c] {\dataFe} ;
    \addlegendentry{Verre calco-sodique}
    \node [fill=white] 
          at (axis description cs:0.78,0.15) {\bfseries \ch{Fe}} ;

  \nextgroupplot[%
    legend pos=outer north east,
    xmin=350, xmax=800,
    minor x tick num=1,
    ymin=1e-20, ymax=2e-17,
  ]
    \pgfplotsset{%
      legend style={nodes={text width=2.3cm, text depth=}}
    }
    \addplot [blue] 
             table [x=lambda_po, y=abs_po] {\dataCr} ;
    \addlegendentry{Verre plombifère oxydé}
    \addplot [blue, dashed] 
             table [x=lambda_pr, y=abs_pr] {\dataCr} ;
    \addlegendentry{Verre plombifère réduit}
    \addplot [red] 
             table [x=lambda_ca, y=abs_ca] {\dataCr} ;
    \addlegendentry{Verre calco-sodique}
    \addplot [red, dashed] 
             table [x=lambda_cb, y=abs_cb] {\dataCr} ;
    \addlegendentry{Verre calco-sodique avec \SI{1}{\percent} de \ch{Sb2O3}}
    \node [fill=white] 
          at (axis description cs:0.22,0.15) {\bfseries \ch{Cr}} ;
  \end{groupplot}
\end{tikzpicture}

















\newpage

\noindent%
\begin{minipage}{0.47\textwidth}
  % \tikz{\draw (0, 0) rectangle (6, 7)}%
  \includegraphics[height=8cm]{Stroud_Fe.png}%

  Absorption causée par le fer dans un verre plombifère 
  (ligne continue) et dans un verre calco-sodique (pointillés) 
  \autocite{Stroud_1971}
\end{minipage}%
\hfill%
\begin{minipage}{0.47\textwidth}
  % \tikz{\draw (0, 0) rectangle (6, 6.5)}%
  \includegraphics[height=8cm]{Stroud_Cu.png}%

  Absorption causée par le cuivre dans un verre plombifère 
  (ligne continue) et dans un verre calco-sodique (pointillés) 
  \autocite{Stroud_1971}
\end{minipage}%

\noindent%
\begin{minipage}{0.47\textwidth}
  % \tikz{\draw (0, 0) rectangle (5.5, 8)}%
  \includegraphics[height=8cm]{Stroud_Cr.png}%

  Absorption causée par le chrome dans un verre plombifère oxydé 
  (ligne continue), dans un verre plombifère réduit (tirets), 
  dans un verre calco-sodique (points et tirets) et dans un verre 
  calco-sodique avec \SI{1}{\percent} de \ch{Sb2O3} (pointillés)
  \autocite{Stroud_1971}
\end{minipage}%
\hfill%
\begin{minipage}{0.47\textwidth}
  % \tikz{\draw (0, 0) rectangle (6.5, 8.5)}%
  \includegraphics[height=8cm]{Stroud_Mn.png}%

  Absorption causée par le manganèse dans un verre plombifère 
  (ligne continue) et dans un verre calco-sodique (pointillés) 
  \autocite{Stroud_1971}
\end{minipage}%

\noindent%
\begin{minipage}{0.47\textwidth}
  % \tikz{\draw (0, 0) rectangle (7.5, 5.5)}%
  \includegraphics[height=8cm]{Stroud_Ni.png}%

  Absorption causée par le nickel dans un verre plombifère 
  (ligne continue) et dans un verre calco-sodique (pointillés) 
  \autocite{Stroud_1971}
\end{minipage}%
