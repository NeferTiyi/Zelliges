%!TEX root = MemoireZelliges.tex

\newcommand{\blackstar}{\ding{88}}
\newcommand{\bluestar}{{\color{blue}\ding{88}}}

\newcommand{\bdx}[1]{BDX\,#1\xspace}
% \newcommand{\PaM}{palais Al-Mansour\xspace}
\newcommand{\PaM}{{\color{red}palais Al-Mansour}\xspace}
\newcommand{\Zell}[2][]{%
  Zellige#1 monochrome#1 #2 provenant du \PaM, \scl{xvii}\xspace
}
\newcommand{\zlaygi}{\emph{zl\={a}y\u{g}\={\i}}\xspace}
\newcommand{\hatim}{\emph{h\={a}tim}\xspace}

\newcommand{\legendeA}{Meknès (Maroc), \bdx{6528}. \Zell{vert}}
\newcommand{\legendeB}{Meknès (Maroc), \bdx{6529}. \Zell{bleu vert}}
\newcommand{\legendeC}{Meknès (Maroc), \bdx{6530}. \Zell{noir}}
\newcommand{\legendeD}{Meknès (Maroc), \bdx{6531}. \Zell{miel}}
\newcommand{\legendeE}{Meknès (Maroc), \bdx{6532}. \Zell{blanc}}

\newcommand{\legendeAll}{%
  Meknès (Maroc), \bdx{6528}, \bdx{6529}, \bdx{6530}, \bdx{6531} et 
  \bdx{6532}. \Zell[s]{vert, bleu, noir, miel et blanc}%
}

\newcommand{\Xn}{\ensuremath{X_\text{n}}}
\newcommand{\Yn}{\ensuremath{Y_\text{n}}}
\newcommand{\Zn}{\ensuremath{Z_\text{n}}}
\newcommand{\CIEL}{\ensuremath{L^\text{*}}\xspace}
\newcommand{\CIEa}{\ensuremath{a^\text{*}}\xspace}
\newcommand{\CIEb}{\ensuremath{b^\text{*}}\xspace}
\newcommand{\CIELb}{\ensuremath{\bm{L^\text{*}}}\xspace}
\newcommand{\CIEab}{\ensuremath{\bm{a^\text{*}}}\xspace}
\newcommand{\CIEbb}{\ensuremath{\bm{b^\text{*}}}\xspace}
\newcommand{\CIEY}{\ensuremath{Y}\xspace}
\newcommand{\CIEx}{\ensuremath{x}\xspace}
\newcommand{\CIEy}{\ensuremath{y}\xspace}
\newcommand{\CIEz}{\ensuremath{z}\xspace}
\newcommand{\CIEYb}{\ensuremath{\bm{Y}}\xspace}
\newcommand{\CIExb}{\ensuremath{\bm{x}}\xspace}
\newcommand{\CIEyb}{\ensuremath{\bm{y}}\xspace}
\newcommand{\CIEzb}{\ensuremath{\bm{z}}\xspace}
\newcommand{\Yxy}{\ensuremath{\CIEY\!\!\CIEx\CIEy}\xspace}
\newcommand{\Lab}{\ensuremath{\CIEL\!\!\CIEa\!\CIEb}\xspace}
\newcommand{\WLd}{\ensuremath{\lambda_\text{d}}\xspace}

\definecolor{RED}{named}{red}

\newcommand{\spectro}{{\color{red}spectrométrie}\xspace}
\newcommand{\carto}{{\color{red}cartographie}\xspace}
\newcommand{\trichro}{{\color{red}trichromatique}\xspace}
\newcommand{\trichros}{{\color{red}trichromatiques}\xspace}
\newcommand{\cristallo}{{\color{red}cristallographique}\xspace}
\newcommand{\cristallos}{{\color{red}cristallographiques}\xspace}

\newcommand{\RX}{{\color{red}rayons~X}\xspace}
\newcommand{\DX}{{\color{red}diffraction de \RX}\xspace}
\newcommand{\MEB}[1][e]{{\color{red}microscop#1 électronique à balayage}\xspace}
\newcommand{\CL}{{\color{red}cathodoluminescence}\xspace}
\newcommand{\AO}{{\color{red}absorption optique}\xspace}
\newcommand{\SAO}{{\color{red}\spectro d'\AO}\xspace}
\newcommand{\ERD}{{\color{red}électrons rétrodiffusés}\xspace}
\newcommand{\EDS}{{\color{red}\spectro de \RX en dispersion d'énergie}\xspace}
\newcommand{\CHRO}{{\color{red}chromamétrie}\xspace}

\newcommand{\DimText}{%
  \emph{Dimensions} (L\,\texttimes\,l\,\texttimes\,e)\xspace%
}

\newcommand{\PMO}{{\color{red}\si{\percent} massique d'oxyde}\xspace}
\newcommand{\zone}[2]{{\color{red}zone analysée : \SI{#1}{#2}}\xspace}

\newcommand{\sclnum}[1]{\bsc{#1}\ieme}
\newcommand{\scl}[1]{\sclnum{#1}~s.}
\renewcommand{\siecle}[1]{\sclnum{#1}~siècle}
