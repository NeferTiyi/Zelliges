%!TEX root = MemoireZelliges.tex

% Numéros de siècles
% ==================
\newcommand{\sieclenum}[1]{%
  \ifrmnum{#1}{%
    \bsc{\MakeLowercase{#1}}%
    \ifstrequal{#1}{i}{\ier}{\ieme}%
  }{%
    \bsc{\MakeLowercase{\romannumeral#1}}%
    \ifnumequal{#1}{1}{\ier}{\ieme}%
  }%
}
\newcommand{\siecle}[2][siècle]{%
  \sieclenum{#2}~#1%
}
\newcommand{\sieclelist}[2]{%
  \sieclenum{#1} et \sieclenum{#2}~siècles%
}
\newcommand{\sieclerange}[2]{%
  \sieclenum{#1}-\sieclenum{#2}~siècles%
}

% % Incise
% % ======
% \DeclareRobustCommand{\oinc}{\textemdash~}
% \DeclareRobustCommand{\finc}{~\textemdash}
% \DeclareRobustCommand{\incise}[1]{\oinc#1\finc}
% \DeclareRobustCommand{\oincise}[1]{\oinc#1}

% Coloration en cas de brouillon
% ==============================
\providecommand{\colordraft}[2]{%
  \texorpdfstring{%
    \ifbool{draft}
           {\hspace*{0pt}{\color{#1}#2}}
           {#2}%
  }{%
    #2%
  }%
}

\newcommand{\blackstar}{\ding{88}}
\newcommand{\bluestar}{{\color{blue}\ding{88}}}

\newcommand{\bdx}[1]{%
  \texorpdfstring{BDX\,#1\xspace}{BDX#1}%
}

\newcommand{\PaM}{\colordraft{red}{Palais al-Mansour}\xspace}
% \newcommand{\PaM}{palais Al-Mansour\xspace}

\newcommand{\Zell}[2][]{%
  Zellige#1 monochrome#1 #2 provenant du \PaM, \siecle{17}\xspace
}

\newcommand{\legendeA}{Meknès (Maroc), \bdx{6528}. \Zell{vert}}
\newcommand{\legendeB}{Meknès (Maroc), \bdx{6529}. \Zell{bleu vert}}
\newcommand{\legendeC}{Meknès (Maroc), \bdx{6530}. \Zell{noir}}
\newcommand{\legendeD}{Meknès (Maroc), \bdx{6531}. \Zell{miel}}
\newcommand{\legendeE}{Meknès (Maroc), \bdx{6532}. \Zell{blanc}}

\newcommand{\legendeAll}{%
  Meknès (Maroc), \bdx{6528}, \bdx{6529}, \bdx{6530}, \bdx{6531} et 
  \bdx{6532}. \Zell[s]{vert, bleu, noir, miel et blanc}%
}

\newcommand{\zlaygi}{\emph{\colordraft{red}{zl\={a}y\v{g}\={\i}}}\xspace}
\newcommand{\hatim}{\emph{\colordraft{red}{h\={a}tim}}\xspace}
\newcommand{\saft}{\emph{\colordraft{red}{\d{s}aft}}\xspace}
\newcommand{\manqas}{\emph{\colordraft{red}{manq\={a}\v{s}}}\xspace}
\newcommand{\naqs}{\emph{\colordraft{red}{naq\v{s}}}\xspace}
\newcommand{\fars}{\emph{\colordraft{red}{far\v{s}}}\xspace}
\newcommand{\tafrisa}{\emph{\colordraft{red}{tafr\={\i}\v{s}a}}\xspace}
\newcommand{\zillig}{\emph{\colordraft{red}{zill\={\i}\v{g}}}\xspace}
\newcommand{\furma}{\emph{\colordraft{red}{f\={u}rma}}\xspace}
\newcommand{\furmas}{\emph{\colordraft{red}{f\={u}rmas}}\xspace}
\newcommand{\farina}{\emph{\colordraft{red}{far\={\i}na}}\xspace}
\newcommand{\mzihrinayy}{\emph{\colordraft{red}{mzihr\={\i} nayy}}\xspace}
\newcommand{\alqala}{\emph{\colordraft{red}{al-q\={a}la}}\xspace}
\newcommand{\tansif}{\emph{\colordraft{red}{tans\={\i}f}}\xspace}
\newcommand{\tagfif}{\emph{\colordraft{red}{ta\v{g}f\={\i}f}}\xspace}
\newcommand{\zwabi}{\emph{\colordraft{red}{zw\={a}b\={\i}}}\xspace}
\newcommand{\maallem}{\emph{\colordraft{red}{maallem}}\xspace}

\newcommand{\Pe}{\ensuremath{P_\text{e}}}
\newcommand{\Xn}{\ensuremath{X_\text{n}}}
\newcommand{\Yn}{\ensuremath{Y_\text{n}}}
\newcommand{\Zn}{\ensuremath{Z_\text{n}}}
\newcommand{\CIEL}{\ensuremath{L^\text{*}}\xspace}
\newcommand{\CIEa}{\ensuremath{a^\text{*}}\xspace}
\newcommand{\CIEb}{\ensuremath{b^\text{*}}\xspace}
\newcommand{\CIELb}{\ensuremath{\bm{L^\text{*}}}\xspace}
\newcommand{\CIEab}{\ensuremath{\bm{a^\text{*}}}\xspace}
\newcommand{\CIEbb}{\ensuremath{\bm{b^\text{*}}}\xspace}
\newcommand{\CIEY}{\ensuremath{Y}\xspace}
\newcommand{\CIEx}{\ensuremath{x}\xspace}
\newcommand{\CIEy}{\ensuremath{y}\xspace}
\newcommand{\CIEz}{\ensuremath{z}\xspace}
\newcommand{\CIEYb}{\ensuremath{\bm{Y}}\xspace}
\newcommand{\CIExb}{\ensuremath{\bm{x}}\xspace}
\newcommand{\CIEyb}{\ensuremath{\bm{y}}\xspace}
\newcommand{\CIEzb}{\ensuremath{\bm{z}}\xspace}
\newcommand{\Yxy}{\ensuremath{\CIEY\!\!\CIEx\CIEy}\xspace}
\newcommand{\Lab}{\ensuremath{\CIEL\!\!\CIEa\!\CIEb}\xspace}
\newcommand{\WLd}{\ensuremath{\lambda_\text{d}}\xspace}
\newcommand{\WLdb}{\ensuremath{\bm{\lambda_\text{d}}}\xspace}

\definecolor{RED}{named}{red}

\newcommand{\spectro}{\colordraft{red}{spectrométrie}\xspace}
\newcommand{\carto}{\colordraft{red}{cartographie}\xspace}
\newcommand{\trichro}{\colordraft{red}{trichromatique}\xspace}
\newcommand{\trichros}{\colordraft{red}{trichromatiques}\xspace}
\newcommand{\cristallo}{\colordraft{red}{cristallographique}\xspace}
\newcommand{\cristallos}{\colordraft{red}{cristallographiques}\xspace}

\newcommand{\RX}{\colordraft{red}{rayons~X}\xspace}
\newcommand{\DX}[1][d]{\colordraft{red}{#1iffraction de \RX{}}\xspace}
\newcommand{\MEB}[1][e]{\colordraft{red}{microscop#1 électronique à balayage}\xspace}
\newcommand{\CL}{\colordraft{red}{cathodoluminescence}\xspace}
\newcommand{\AO}{\colordraft{red}{absorption optique}\xspace}
\newcommand{\SAO}{\colordraft{red}{\spectro d'\AO{}}\xspace}
\newcommand{\ERD}{\colordraft{red}{électrons rétrodiffusés}\xspace}
\newcommand{\EDS}{\colordraft{red}{\spectro de \RX en dispersion d'énergie}\xspace}
\newcommand{\CHRO}{\colordraft{red}{chromamétrie}\xspace}

\newcommand{\DimText}{%
  \emph{Dimensions} (L\,\texttimes\,l\,\texttimes\,e)\xspace%
}

\newcommand{\PMO}{\colordraft{red}{\si{\percent} massique d'oxyde}\xspace}
\newcommand{\zone}[2]{\colordraft{red}{zone analysée : \SI{#1}{#2}}\xspace}

