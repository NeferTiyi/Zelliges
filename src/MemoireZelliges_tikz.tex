%!TEX root = MemoireZelliges.tex

\newcommand{\drawhatim}{%
  (    0.000 , {sqrt(2)}) -- ({1-sqrt(2)},     1.000 ) --
  (   -1.000 ,    1.000 ) -- (    -1.000 ,{sqrt(2)-1}) --
  ({-sqrt(2)},    0.000 ) -- (    -1.000 ,{1-sqrt(2)}) --
  (   -1.000 ,   -1.000 ) -- ({1-sqrt(2)},    -1.000 ) --
  (    0.000 ,-{sqrt(2)}) -- ({sqrt(2)-1},    -1.000 ) --
  (    1.000 ,   -1.000 ) -- (     1.000 ,{1-sqrt(2)}) --
  ( {sqrt(2)},    0.000 ) -- (     1.000 ,{sqrt(2)-1}) --
  (    1.000 ,    1.000 ) -- ({sqrt(2)-1},     1.000 ) -- cycle
}

\newcommand{\drawluza}{%
  (     0.000 ,   0.000 ) -- ({2-sqrt(2)},     0.000 ) -- 
  ({2-sqrt(2)},{sqrt(2)}) -- ({1-sqrt(2)},{sqrt(2)-1}) -- cycle
}

\newcommand{\drawsaft}{%
  (-2.00, 0.00) -- (-1.00, 1.00) -- ( 1.00, 1.00) -- 
  ( 2.00, 0.00) -- ( 1.00,-1.00) -- (-1.00,-1.00) -- cycle
}

% % Fawrkat
% \fill [color=ffqqff, fill=ffqqff, fill opacity=0.1] 
%       ( 0.414,-1.000) -- (-1.000,-1.000) -- 
%       (-2.000, 0.000) -- (-1.000, 1.000) -- 
%       ( 0.414, 1.000) -- ( 0.414, 0.414) -- 
%       ( 0.000, 0.000) -- ( 0.414,-0.414) -- cycle ;

\newcommand{\squeletteb}{%
  \draw [domain=-1:1] plot( \x, {-\x} ) ;
  \draw [domain=-2:2] plot( 0, \x ) ;
  \draw [domain=0:{2-2*sqrt(2)}] plot(-2, \x ) ;
  \draw [domain=0:{2*sqrt(2)-2}] plot( 2, \x ) ;
  \draw [domain=-2:2] plot( \x, {+sqrt(2)} ) ;
  \draw [domain=-2:2] plot( \x, {-sqrt(2)}   ) ;
  \draw [domain=0:5]  plot( \x, {+sqrt(2)+2} ) ;
  \draw [domain=-5:0] plot( \x, {-sqrt(2)-2}   ) ;

  \draw [domain=2:1] plot(\x, {-\x + 2*sqrt(2)}) ;
  \draw [domain=-2:-1] plot(\x, {-\x - 2*sqrt(2)}) ;

  \draw [domain=-1:-3] plot( \x, {\x + 2}) ;
  \draw [domain=1:3] plot( \x, {\x - 2}) ;
  \draw [domain=-6:-2] plot( \x, {\x + 2*sqrt(2) + 2}) ;
  \draw [domain=2:6] plot( \x, {\x - 2*sqrt(2) - 2}) ;

  \filldraw [shift={( 1 , {-sqrt(2)-1} )}] 
    \drawhatim ;
  \filldraw [shift={( -1 , {sqrt(2)+1} )}] 
    \drawhatim ;
  \filldraw [shift={( {-2*sqrt(2)-3} , {-sqrt(2)-1} )}] 
    \drawhatim ;
  \filldraw [shift={( {2*sqrt(2)+3} , {sqrt(2)+1} )}] 
    \drawhatim ;

  \filldraw [shift={( {-sqrt(2)-1} , {-sqrt(2)-1} )}] 
    \drawsaft ;
  \filldraw [shift={( {sqrt(2)+1} , {sqrt(2)+1} )}] 
    \drawsaft ;
  \filldraw [shift={( {-sqrt(2)-2} , 0 )}, rotate=45] 
    \drawsaft ;
  \filldraw [shift={( {sqrt(2)+2} , 0 )}, rotate=45] 
    \drawsaft ;
}

\newcommand{\squelettea}{%
  \begin{scope}
    \clip (-4,-4) rectangle (4,4) ;
    \draw plot( \x, {\x + sqrt(2)}) ;
    \draw plot( \x, {\x - sqrt(2)}) ;
    \draw plot(-\x, {\x + sqrt(2)}) ;
    \draw plot(-\x, {\x - sqrt(2)}) ;
  \end{scope}

  \draw ( {-1-sqrt(2)} , {-1-sqrt(2)} ) rectangle
        ( {+1+sqrt(2)} , {+1+sqrt(2)} ) ;
  \draw ( {-3-sqrt(2)} , {-3-sqrt(2)} ) rectangle
        ( {+3+sqrt(2)} , {+3+sqrt(2)} ) ;

  \filldraw [shift={( {+2+sqrt(2)} , {+2+sqrt(2)} )}] 
    \drawhatim ;
  \filldraw [shift={( {+2+sqrt(2)} , {-2-sqrt(2)} )}] 
    \drawhatim ;
  \filldraw [shift={( {-2-sqrt(2)} , {+2+sqrt(2)} )}] 
    \drawhatim ;
  \filldraw [shift={( {-2-sqrt(2)} , {-2-sqrt(2)} )}] 
    \drawhatim ;

  \filldraw [shift={( 0 , {-2-sqrt(2)} )}] 
    \drawsaft ;
  \filldraw [shift={( 0 , {+2+sqrt(2)} )}] 
    \drawsaft ;
  \filldraw [shift={( {-2-sqrt(2)} , 0 )} , rotate=90] 
    \drawsaft ;
  \filldraw [shift={( {+2+sqrt(2)} , 0 )} , rotate=90] 
    \drawsaft ;
}

\newcommand{\drawfluoX}{%
  ( 0.19,-0.32) .. controls ( 0.22,-0.40) and ( 0.23,-0.43) ..
  ( 0.50,-0.79) .. controls ( 0.60,-0.95) and ( 0.66,-1.12) ..
  ( 0.66,-1.32) .. controls ( 0.65,-1.78) and ( 0.30,-1.98) ..
  ( 0.00,-1.98) .. controls (-0.30,-1.98) and (-0.65,-1.78) ..
  (-0.66,-1.32) .. controls (-0.65,-1.12) and (-0.60,-0.95) ..
  (-0.50,-0.79) .. controls (-0.23,-0.43) and (-0.22,-0.40) ..
  (-0.19,-0.32)
}

\newcommand{\drawRX}{%
  ( 0.15,-0.06) to [out=-87, in=104] 
  ( 0.19,-0.32) .. controls ( 0.21,-0.40) and ( 0.23,-0.46) ..
  ( 0.26,-0.51) .. controls ( 0.31,-0.60) and ( 0.35,-0.65) ..
  ( 0.41,-0.74) .. controls ( 0.57,-0.99) and ( 0.66,-1.30) ..
  ( 0.50,-1.54) .. controls ( 0.39,-1.71) and ( 0.24,-1.81) ..
  ( 0.00,-1.81) .. controls (-0.24,-1.81) and (-0.39,-1.71) ..
  (-0.50,-1.54) .. controls (-0.66,-1.30) and (-0.57,-0.99) ..
  (-0.41,-0.74) .. controls (-0.35,-0.65) and (-0.31,-0.60) ..
  (-0.26,-0.51) .. controls (-0.23,-0.46) and (-0.21,-0.40) ..
  (-0.19,-0.32) to [out=76, in=-93] 
  (-0.15,-0.06)
}

\newcommand{\drawinterne}{%
  ( 0.11,-0.06) .. controls ( 0.13,-0.46) and ( 0.23,-0.53) .. 
  ( 0.34,-0.71) .. controls ( 0.50,-0.96) and ( 0.56,-1.21) .. 
  ( 0.42,-1.43) .. controls ( 0.32,-1.59) and ( 0.19,-1.65) .. 
  ( 0.00,-1.65) .. controls (-0.19,-1.65) and (-0.32,-1.59) .. 
  (-0.42,-1.43) .. controls (-0.56,-1.21) and (-0.50,-0.96) .. 
  (-0.34,-0.71) .. controls (-0.23,-0.53) and (-0.13,-0.46) .. 
  (-0.11,-0.06)
}

\newcommand{\drawaugier}{%
  (-0.15, 0.00) rectangle ( 0.15, -0.06)
}

\newcommand{\MEBpoire}{%
  \begin{scope}[
    scale=2.3,
    every node/.style={font=\scriptsize},
    line cap=round,
  ]

    \pgfmathsetmacro{\xA}{0.145}
    \pgfmathsetmacro{\yA}{-0.35}
    \pgfmathsetmacro{\xB}{0.495}
    \pgfmathsetmacro{\yB}{-1.11}
    \coordinate (A1) at (-\xA, \yA) ;
    \coordinate (A2) at ( \xA, \yA) ;
    \coordinate (B1) at (-\xB, \yB) ;
    \coordinate (B2) at ( \xB, \yB) ;

    % .. Remplissage ..
    % ... Fluo X ...
    \fill [DarkSlateBlue!20!white] \drawfluoX ;
    % ... RX ...
    \fill [DarkSlateBlue!30!white] \drawRX ;
    % ... RX carac ...
    \fill [DarkSlateBlue!40!white] \drawinterne ;

    \begin{scope}
      \clip \drawinterne ;
      % ... Secondaires ...
      \fill [DarkSlateBlue!50!white] 
            (-\xB, \yB) rectangle ( \xB, \yA) ;
      % ... Rétrodiff ...
      \fill [DarkSlateBlue!60!white] 
            (-\xA, \yA) rectangle ( \xA, 0.00) ;
    \end{scope}
    % Électrons Augier
    \fill [fill=DarkSlateBlue!70!white] \drawaugier ;

    % .. Contours ..
    \begin{scope}[DarkSlateBlue]
      % ... Fluo X ...
      \draw \drawfluoX ;
      % ... RX ...
      \draw \drawRX ;
      % ... RX carac ...
      \draw \drawinterne ;
      % Limite secondaires / rétrodiff
      \draw (A1) -- (A2) ;
      % limite rétrodiff / RC carac
      \draw (B1) -- (B2) ;
      % Électrons Augier
      \draw \drawaugier ;
    \end{scope}

    % .. Faisceau incident ..
    \filldraw [draw=DarkRed, bottom color=DarkRed, top color=white]
          (-0.30, 0.77) -- (-0.15, 0.00) -- ( 0.00,  0.00) -- 
          ( 0.15, 0.00) -- ( 0.30, 0.77) 
          node (G) {} ;

    % .. Surface échantillon ..
    \draw [thick]
          (-1.40, 0.00) -- ( 1.40, 0.00) 
          node (H) [pos=1] {} ;

    % .. Légende ..
    \coordinate (A1) at (-0.10,-0.03) ;
    \coordinate (A2) at (-0.50,-0.15) ;
    \coordinate (B1) at (-0.05,-0.20) ;
    \coordinate (B2) at (-0.50,-0.45) ;
    \coordinate (C1) at ( 0.10,-0.80) ;
    \coordinate (C2) at ( 0.80,-0.80) ;
    \coordinate (D1) at ( 0.20,-1.30) ;
    \coordinate (D2) at ( 0.80,-1.30) ;
    \coordinate (E1) at (-0.50,-1.40) ;
    \coordinate (E2) at (-0.80,-1.40) ;
    \coordinate (F1) at (-0.50,-1.65) ;
    \coordinate (F2) at (-0.80,-1.75) ;
  \end{scope}
}

\newcommand{\MEBlegende}{%
  % .. Légende ..
  \begin{scope}[every node/.style={font=\smaller}]
    \draw (A1) -- (A2) 
          node [left] {électrons Auger} ;
    \draw (B1) -- (B2) 
          node [left, align=right] {électrons\\secondaires} ;
    \draw (C1) -- (C2) 
          node [right, align=left] {électrons\\rétrodiffusés} ;
    \draw (D1) -- (D2) 
          node [right, align=left] {\RX\\caractéristiques} ;
    \draw (E1) -- (E2) 
          node [left, align=right] {continuum\\de \RX} ;
    \draw (F1) -- (F2) 
          node [left] {fluorescence~X} ;
    \node [below right, align=left] at (G) 
          {faisceau incident\\d'électrons} ;
    \node [right, align=left] at (H) {surface de\\l'échantillon} ;
  \end{scope}
}

\newcommand{\CLapp}{%
  \begin{scope}[
    scale=0.1, 
    rounded corners=0.5, 
    >=stealth', 
    thick,
  ]
    % \draw [help lines, step=1, ultra thin, densely dotted] 
    %       (-40.00,-10.00) grid ( 40.00, 60.00) ;
    % \draw [help lines, step=5, ultra thin] 
    %       (-40.00,-10.00) grid ( 40.00, 60.00) ;

    \begin{scope}
      % Moniteur
      \draw ( 0.00,  0.00) rectangle ++ ( 30.00, 24.00) ;
      \draw ( 2.00,  8.00) rectangle ++ ( 26.00, 14.00) ;
      \draw ( 2.00,  2.00) rectangle ++ (  2.00,  2.00) ;
      % Système d'acquisition
      \draw ( 0.00, 28.00) rectangle ++ ( 30.00,  6.00) ;
      \draw ( 0.00, 38.00) rectangle ++ ( 30.00,  6.00) ;
      \draw ( 0.00, 48.00) rectangle ++ ( 30.00,  6.00) ;
    \end{scope}

    \begin{scope}[shift={(-15.00, 20.00)}]
      % Loupe bino
      \draw ( 2.00,  0.00) -- ( 5.80,  4.80) -- ( 5.80, 16.20) --
            ( 2.90, 16.20) -- ( 2.90, 21.20) -- ( 3.10, 21.30) -- 
            ( 3.10, 22.90) -- ( 3.70, 23.40) -- ( 3.70, 24.10) -- 
            (-3.70, 24.10) -- (-3.70, 23.40) -- (-3.10, 22.90) -- 
            (-3.10, 21.30) -- (-2.90, 21.20) -- (-2.90, 16.20) -- 
            (-5.80, 16.20) -- (-5.80,  4.80) -- (-2.00,  0.00) -- 
            cycle
         ;
      \draw (-2.00, 0.00) -- (-1.00, 1.00) -- 
            ( 1.00, 1.00) -- ( 2.00, 0.00) ;
      \draw (-2.90, 21.20) -- ( 2.90, 21.20) ;
      \draw (-3.10, 22.90) -- ( 3.10, 22.90) ;
      \draw (-2.90, 18.70) arc (113.4:66.6:7.3) ;
      \draw (-2.90, 18.70) arc (-113.4:-66.6:7.3) ;

      % Caméra CCD
      \draw [shift={( 0.00, 30.00)}, pattern=north west lines]
            ( 4.00,-5.90) -- ( 4.00,-3.50) -- ( 6.00,-3.50) -- 
            ( 6.00, 3.50) -- (-1.60, 3.50) -- (-1.60, 4.80) -- 
            (-3.60, 4.80) -- (-3.60, 3.50) -- (-5.70, 3.50) -- 
            (-7.00, 0.70) -- (-7.00,-0.70) -- (-5.70,-3.50) -- 
            (-4.00,-3.50) -- (-4.00,-5.90) -- cycle
         ;
    \end{scope}

    % Chambre Nuclide
    \begin{scope}[shift={(-27.00, 7.00)}]
      \draw ( 0.00, 0.00) rectangle ++ ( 20.00, 8.00) ;
      \filldraw ( 0.00, 5.00) arc (90:-90:1) ;

      \begin{scope}[shift={( 9.00, 3.25)}]
        \draw ( 0.00, 0.00) rectangle ++ ( 3.00, 1.50) ;
        \draw [->, decorate, decoration={snake, 
               amplitude=.5mm, segment length=2mm, post length=2mm}]
              ( 1.50, 1.5) -- ++ ( 1.50, 8.25) ;
      \end{scope}

      \draw [->] 
            ( 2.00, 4.00) -- ( 7.00, 4.00) 
            node [midway, above] {$e^-$} ;
    \end{scope}

    % Liens
    \draw ( 15.00, 24.00) -- ++ ( 0.00, 4.00) ;
    \draw ( 15.00, 34.00) -- ++ ( 0.00, 4.00) ;
    \draw ( 15.00, 44.00) -- ++ ( 0.00, 4.00) ;
    \draw ( 15.00, 54.00) -- ++ ( 0.00, 4.00) -- ++ 
          (-30.00,  0.00) -- ++ ( 0.00,-4.50) ;

    % % Légende
    \coordinate (A) at (-28.00, 11.00) ;
    \coordinate (B) at (-22.00, 30.00) ;
    \coordinate (C) at (-23.00, 50.00) ;
    \coordinate (D) at ( 31.00, 11.00) ;
    \coordinate (E) at ( 32.00, 54.00) ;
    \coordinate (F) at ( 32.00, 28.00) ;
  \end{scope}
}

\newcommand{\CLappL}{%
  % Légende
  \begin{scope}[every node/.style={font=\smaller}]
    \node [left, align=right] at (A) {chambre\\Nuclide} ;
    \node [left, align=right] at (B) {loupe\\binoculaire} ;
    \node [left, align=right] at (C) {caméra\\3CCD} ;
    \node [right, align=left] at (D) {moniteur} ;

    \draw [decorate, decoration=brace] (E) -- (F)
          node [midway, right, align=left] 
               {système d'acquisition\\et de traitement} ;
  \end{scope}
}

\newcommand{\CLpcp}{%
  \begin{scope}[>=stealth', every node/.style={font=\smaller}]

    % \draw [help lines, densely dotted, red, step=1] 
    %       (-10.00, -20.00) grid ( 60.00, 50.00) ;
    % \draw [help lines, ultra thin, red, step=5] 
    %       (-10.00, -20.00) grid ( 60.00, 50.00) ;

    \begin{scope}[every node/.style={font=\smaller}, darkgray]
      \draw [->] 
            (-0.20,-1.00) -- (-0.20, 5.50) 
            node [below right, align=left] {énergie\\relative} ;
      \node at (-0.80, 4.20) {BC} ;
      \node at (-0.80, 1.75) {BI} ;
      \node at (-0.80,-0.70) {BV} ;
    \end{scope}

    \begin{scope}[thick, pattern=north east lines]
      \filldraw ( 0.50,-0.50) circle (0.20) ;
      \filldraw ( 4.20,-0.50) circle (0.20) ;
    \end{scope}

    \begin{scope}[thick]
      \draw (-0.70, 0.00) -- ( 5.00, 0.00) ;
      \draw (-0.70, 3.50) -- ( 1.20, 3.50) -- 
            ( 1.20, 1.75) -- ( 2.00, 1.75) -- 
            ( 2.00, 3.50) -- ( 5.00, 3.50) ; 
      \draw ( 3.20, 2.50) -- ( 4.20, 2.50) 
            node [right] {\ce{(I+)^*}} ;
      \draw ( 2.50, 1.00) -- ( 4.20, 1.00) 
            node [right] {\ce{(I+)}} ;
    \end{scope}

    \begin{scope}[->, densely dotted, thick, DarkSlateBlue]
      \draw ( 0.50,-0.30) -- ( 0.50, 3.80) -- 
            ( 2.70, 3.80) node [midway, above] {$a$} --
            ( 2.70, 1.00) node [pos=0.33, left]  {$b$} ;
      \draw ( 3.40, 1.00) -- ( 3.40, 2.50) 
            node [pos=0.33, left] {$c$} ;
      \draw ( 4.00, 2.50) -- ( 4.00, 1.00) 
            node [pos=0.67, right] {$d$} ;
      \draw ( 4.00, 1.00) -- ( 4.20,-0.30) 
            node [pos=0.45, left] {$e$} ;
    \end{scope}

    \draw [
      shift={( 4.20, 2.00)},
      DarkRed, ->, 
      decorate, decoration={snake, amplitude=.5mm, segment length=2mm,
                            pre length=0.5mm, post length=0.5mm}
    ]     ( 0.00, 0.00) -- ( 1.00,  0.00)
          node [right] {$h\nu$} ;

    \draw [DarkRed, ->, ultra thick]
          ( 1.00,-1.50) -- ( 0.60,-0.70) 
          node [pos=0, right, align=left] 
               {bombardement\\électronique} ;

    \coordinate (L) at ( 6.00, 0.00) ;
  \end{scope}
}

\newcommand{\CLpcpL}{%
  \begin{scope}[every node/.style={font=\smaller}]
    \node at (L) [above right, align=left, text width=6.3cm] 
    {
      \begin{itemize}
        \item [BC :] bande de conduction ;
        \item [BI :] bande interdite ;
        \item [BV :] bande de valence.
      \end{itemize}

      \bigskip

      \begin{itemize}
        \item [$a$ :] ionisation d'un atome sans intervention 
              d'un centre piège ;
        \item [$b$ :] capture de l'électron par l'impureté 
              \ce{(I+)} ;
        \item [$c$ :] excitation en \ce{(I+)^*} ;
        \item [$d$ :] désexcitation radiative ;
        \item [$e$ :] recapture de l'électron dans la bande 
              de valence (il n'y a pas forcément ionisation, 
              l'électron peut directement être capturé par 
              l'impureté).
      \end{itemize}
    } ;
  \end{scope}
}

\newcommand{\DXapp}{%
  \begin{scope}[
    % handle active characters in code,
    >=stealth',
    every node/.style={font=\small},
    line cap=round
  ]
    \pgfmathsetmacro{\r}{3.00}
    \pgfmathsetmacro{\a}{20.00}

    \coordinate (O) at ( 0.00, 0.00) ;

    % Cercle gonio
    \draw (O) circle (\r) ;

    % Légende I
    \begin{scope}[DarkSlateBlue]
      \draw [->, shift={(90-\a:0.15*\r)}, rotate=-\a]
            (-0.1*\r, 0.00) arc (170:10:0.1*\r) 
            node [midway, right] {$\uline{\omega}$} ;
      \draw [->] (O) (-2*\a:1.30*\r) arc (-2*\a:-3*\a:1.30*\r) 
            node [above] {$\uline{2\omega}$} ;

      \draw (-0.27*\r, 0.00) arc (180:180-\a:0.27*\r) 
            node [midway, left] {$\theta$} ;
      \draw [rotate=-\a] 
            ( 0.27*\r, 0.00) arc (0:-\a:0.27*\r) ;
      \draw ( 0.47*\r, 0.00) arc (0:-2*\a:0.47*\r) 
            node [pos=0.25, right] {$\theta$} 
            node [pos=0.80, right] {$\theta$} ;
      \draw ( 0.76*\r, 0.00) arc (0:-2*\a:0.76*\r) 
            node [midway, right] {$2\theta$} ;
    \end{scope}

    % Trajet linéaire
    \draw [dashed] (-\r, 0.00) -- ( \r, 0.00) ;

    % Échantillon
    \begin{scope}[rotate=-\a]
      \draw (-0.50*\r, 0.00) -- ( 0.50*\r, 0.00) ;
      \draw [yshift=0.4pt, thick, pattern=north east lines, 
             preaction={fill=white}]
            ( 0.33*\r, 0.00) rectangle (-0.33*\r, 0.10*\r) 
            node (E) [pos=0.85] {} ;
    \end{scope}

    % Trajet rayons
    \begin{scope}[thick, DarkRed, line cap=round]
      \draw [postaction={decorate}, decoration={markings, 
             mark=at position 0.15 with {\arrow[scale=1.4]{>};}}]
            (-1.3*\r, 0.00) -- (O) 
            node [midway, below] {RX incident} ;
      \draw [postaction={decorate}, decoration={markings, 
             mark=at position 0.90 with {\arrow[scale=1.4]{>};}}]
            (O) -- (-2*\a:\r) 
            node [pos=0.67, left, align=left] {RX\\diffracté} ;
    \end{scope}

    % Détecteur
    \begin{scope} [
      thick,
      rounded corners=1,
      shift={(-2*\a:\r)},
      rotate=-90-2*\a,
      fill=white,
    ]
      \filldraw (-0.09*\r, 0.00) rectangle ( 0.09*\r, 0.10*\r) ;
      \filldraw (-0.12*\r, 0.10*\r) rectangle ( 0.12*\r, 0.40*\r) 
            node (D) [pos=0.45] {} ;
    \end{scope}

    % Fente
    \begin{scope}
      \shorthandoff{;}
      \tikzmath{
        \e = 0.03*\r ;          % épaisseur
        \l = 0.19*\a ;          % largeur angulaire
        \h = \r ;               % rayon de courbure
        \y = \h * sin(\l) ;
        \x = \h * (1-cos(\l)) ;
        \b = 4.90 ;
      }
      \shorthandon{;}
      \foreach \s in {-1, 1} {%
        \filldraw [rotate=-2*\a+\s*\b, shift={( \r, 0.00)}]
              (-\x, -\y)  arc (-\l: \l:\h) -- ++ 
              ( \e, 0.00) node (F) [pos=0.5] {} 
               arc ( \l:-\l:\h) -- cycle ;
      }
    \end{scope}

    % Ampli / enregistreur
    \begin{scope}[shift={( 6.00, 1.00)}, rounded corners=1]
      \begin{scope}[thick]
        \draw ( 0.00, 0.00) rectangle ++ ( 6.00,-4.00) 
              node (G) [midway, shift={( 0.00, 1.50)}] {} ;
        \draw ( 0.00, 0.00) rectangle ++ ( 6.00, 1.40) 
              node (A) [midway] {} ;
      \end{scope}
      \begin{scope}[->]
        \draw ( 0.50,-3.50) -- ++ ( 5.00, 0.00) 
              node [below] {\scriptsize$2\theta$} ;
        \draw ( 0.50,-3.50) -- ++ ( 0.00, 2.50) 
              node [above] {\scriptsize$\uline{T}$} ;
      \end{scope}

      % Diffractogramme
      \begin{scope}[shift={( 0.50,-3.30)}, x=0.55mm, y=0.4mm]
        \draw (  0.06,  0.63) .. controls (  4.85,  0.20) and (  3.95,  6.14) .. 
              (  4.35,  9.27) -- 
              (  5.13, 36.84) -- 
              (  6.20,  2.81) .. controls (  7.37, -1.02) and ( 12.72,  0.93) .. 
              ( 15.76,  0.78) .. controls ( 19.59,  1.22) and ( 19.37,  5.78) .. 
              ( 19.79,  8.65) -- 
              ( 20.99, 32.80) -- 
              ( 22.78,  5.59) .. controls ( 23.20,  0.50) and ( 28.08,  0.44) .. 
              ( 31.76,  0.41) .. controls ( 39.54,  0.61) and ( 47.31,  0.40) .. 
              ( 55.09,  0.37) .. controls ( 58.72,  0.50) and ( 61.02,  0.87) .. 
              ( 62.55,  5.38) -- 
              ( 63.48, 45.63) -- 
              ( 63.61, 53.65) -- 
              ( 64.34, 20.19) .. controls ( 64.60, 13.97) and ( 64.80,  7.61) .. 
              ( 65.40,  1.52) .. controls ( 68.18, -0.46) and ( 71.96,  0.68) .. 
              ( 75.15,  0.64) .. controls ( 79.53,  0.10) and ( 78.66,  5.75) .. 
              ( 79.21,  8.77) -- 
              ( 79.65, 22.73) -- 
              ( 80.35, 11.69) .. controls ( 80.50, 10.18) and ( 81.02,  6.56) .. 
              ( 81.36,  5.14) -- 
              ( 82.98, 23.23) -- 
              ( 83.55, 23.99) -- 
              ( 84.85,  5.20) .. controls ( 85.35,  1.09) and ( 87.93,  1.12) .. 
              ( 91.11,  0.00) ;

        \node (P1) at (  5.13, 3.00) {} ;
        \node (P2) at ( 20.99, 3.00) {} ;
      \end{scope}
    \end{scope}

    % Liens
    \draw ( 0.00, 0.00) (-2*\a:1.4*\r) -- ++ 
          ( 2.30, 0.00) -- ++ ( 0.00, 4.40) -- ( 6.00, 1.70) ;

    % Légende II
    \begin{scope}[Gray!25!DarkGray]
      \shorthandoff{;}
      \tikzmath{
        \x = \r*cos(45) ;
        \y = \r*sin(45) ;
      }
      \shorthandon{;}
      \node at (\x, \y) 
            [right, shift={( 0.20, 0.00)}, align=left] 
            {cercle\\goniométrique} ;

      \node at (A) {Ampli} ;
      \node at (G) {Enregistreur} ;

      \draw (E.center) -- ++ (-0.08*\r, 0.23*\r) 
            node [above] {échantillon} ;

      \draw (D.center) -- ++ ( 0.20*\r, 0.10*\r) 
            node [right] {détecteur} ;

      \draw (F.center) -- ++ ( 0.20*\r, 0.10*\r) 
            node [right] {fente} ;

      \draw (P1.center) -- ++ ( 0.05*\r, -0.32*\r) -- (P2.center)
            node [pos=0, below] {pics de diffraction} ;
    \end{scope}
  \end{scope}
}

\newcommand{\DXpcp}{%
  \tikzfading[
    name=fadeout,
    inner color=transparent!0,
    outer color=transparent!100,
  ]

  \tikzset{
    onde/.style={
      thick,
      line cap=round,
      decorate,
      decoration={
        snake,
        mirror,
        amplitude=.7mm,
        segment length=2.5mm,
        % segment length=5mm,
        % pre length=0.31mm,
        post length=1.5mm,
      }
    }
  }

  \begin{scope}[
    >=stealth', 
    line cap=round, 
  ]

    \coordinate (O) at ( 0.00, 0.00) ;
    \pgfmathsetmacro{\xm}{4.00}    % x min/max
    \pgfmathsetmacro{\t}{30.00}    % theta
    \pgfmathsetmacro{\d}{1.00}     % d

    % \draw [help lines, densely dotted, red, step=0.1] 
    %       (-\xm, 0.00) grid ( \xm, 5.00) ;
    % \draw [help lines, ultra thin, red, step=0.5] 
    %       (-\xm, 0.00) grid ( \xm, 5.00) ;
    % \draw [help lines, very thin, blue] 
    %       ( 0.00, 0.00) -- ( 0.00, 5.00) ;

    \foreach \y in {0.00, 1.00, 2.00} {
      \draw (-\xm, \y) -- ( \xm, \y) ;
      \foreach \x in {-3.00, -1.50, ..., 4.00} {
        \fill [DarkSlateBlue, path fading=fadeout] 
              (\x, \y) circle (0.2) ;
      }
    }

    \shorthandoff{;}
    \tikzmath{
      \x = \xm ;
      \y = \x * tan(\t) ;
      \l = 0.25 ;     % longueur d'onde
      % \l = \x / cos(\t) ;
      \b = \d * cos(\t) ;
    }
    \shorthandon{;}

    \foreach \o in {0.00, 1.00} {
      \begin{scope}[every path/.style={onde}]
        \draw [->] 
              ( 0.00, \o)  -- ( \x, \y+\o) ;
        \draw [-<, decoration={mirror=false}] 
              ( 0.00, \o) -- (-\x, \y+\o) ;
      \end{scope}

      \draw [very thin] 
            (-\x, \y+\o) -- ( 0.00, \o) -- ( \x, \y+\o) ;
    }

    \begin{scope}[DarkRed]
      \draw ( 0.00, \d) -- (O) ;
      \foreach \s in {-1, 1} {
        \foreach \o in {0, 4.75} {
          \draw [xscale=\s]
                ( 0.00, \d) ++ (\t:0.015+\o*\l)     -- ++ (-90+\t:\b) ;
        }
        \draw [xscale=\s, very thin]
              ( 0.50, \d) arc (0:\t:0.5) ;
        \node at ([xscale=\s] 0.60, 1.12) {\smaller$\theta$} ;
        \draw [xscale=\s, very thin]
              ( 0.00, 0.50) arc (-90:-89+\t:0.5) ;
        \node at ([xscale=\s] 0.18, 0.40) {\smaller$\theta$} ;
      }

      \draw [|-|, shift={(180-\t:0.015+12.75*\l)}]
        ([shift={(90-\t:0.15)}] 0.00, \d) -- ++ (180-\t:1*\l) 
        node [pos=0, above] {\smaller$\lambda$} ;

      \draw [<->, shorten >=1pt, shorten <=1pt, ] 
            (\xm-0.2, 0.00) -- ++ ( 0.00, \d) 
            node [midway, right] {\smaller$d$} ;
    \end{scope}

  \end{scope}
}
