%!TEX root = MemoireZelliges.tex
\newcommand{\definelength}{%
  \pgfmathsetmacro{\rayonH}{sqrt(2)}
  \pgfmathsetmacro{\lenH}{(sqrt(2)-1)/cos(45)}
}

\newcommand{\drawhatim}{%
  % (    0.000 , {sqrt(2)}) -- ({1-sqrt(2)},     1.000 ) --
  % (   -1.000 ,    1.000 ) -- (    -1.000 ,{sqrt(2)-1}) --
  % ({-sqrt(2)},    0.000 ) -- (    -1.000 ,{1-sqrt(2)}) --
  % (   -1.000 ,   -1.000 ) -- ({1-sqrt(2)},    -1.000 ) --
  % (    0.000 ,-{sqrt(2)}) -- ({sqrt(2)-1},    -1.000 ) --
  % (    1.000 ,   -1.000 ) -- (     1.000 ,{1-sqrt(2)}) --
  % ( {sqrt(2)},    0.000 ) -- (     1.000 ,{sqrt(2)-1}) --
  % (    1.000 ,    1.000 ) -- ({sqrt(2)-1},     1.000 ) -- cycle
    ++ (90:\rayonH) 
  --++ (225:\lenH) 
  --++ (180:\lenH) 
  --++ (-90:\lenH) 
  --++ (225:\lenH) 
  --++ (-45:\lenH) 
  --++ (-90:\lenH) 
  --++ (  0:\lenH) 
  --++ (-45:\lenH) 
  --++ ( 45:\lenH) 
  --++ (  0:\lenH) 
  --++ ( 90:\lenH) 
  --++ ( 45:\lenH) 
  --++ (135:\lenH) 
  --++ ( 90:\lenH) 
  --++ (180:\lenH) 
  -- cycle
}

% \newcommand{\drawluza}{%
%   (     0.000 ,   0.000 ) -- ({2-sqrt(2)},     0.000 ) -- 
%   ({2-sqrt(2)},{sqrt(2)}) -- ({1-sqrt(2)},{sqrt(2)-1}) -- cycle
% }

\newcommand{\drawsaft}{%
  % (-2.00, 0.00) -- (-1.00, 1.00) -- ( 1.00, 1.00) -- 
  % ( 2.00, 0.00) -- ( 1.00,-1.00) -- (-1.00,-1.00) -- cycle
       ( 2.00, 0.00)
  --++ ( 135.0:\rayonH)
  --++ ( 180.0:2)
  --++ (-135.0:\rayonH)
  --++ ( -45.0:\rayonH)
  --++ (   0.0:2)
  -- cycle
}

% % Fawrkat
% \fill [color=ffqqff, fill=ffqqff, fill opacity=0.1] 
%       ( 0.414,-1.000) -- (-1.000,-1.000) -- 
%       (-2.000, 0.000) -- (-1.000, 1.000) -- 
%       ( 0.414, 1.000) -- ( 0.414, 0.414) -- 
%       ( 0.000, 0.000) -- ( 0.414,-0.414) -- cycle ;

\newcommand{\squeletteb}{%
  \definelength

  \draw [domain=-1:1] plot( \x, {-\x} ) ;
  \draw [domain=-2:2] plot( 0, \x ) ;
  \draw [domain=0:{2-2*\rayonH}] plot(-2, \x ) ;
  \draw [domain=0:{2*\rayonH-2}] plot( 2, \x ) ;
  \draw [domain=-2:2] plot( \x, {+\rayonH} ) ;
  \draw [domain=-2:2] plot( \x, {-\rayonH}   ) ;
  \draw [domain=0:5]  plot( \x, {+\rayonH+2} ) ;
  \draw [domain=-5:0] plot( \x, {-\rayonH-2}   ) ;

  \draw [domain=2:1] plot(\x, {-\x + 2*\rayonH}) ;
  \draw [domain=-2:-1] plot(\x, {-\x - 2*\rayonH}) ;

  \draw [domain=-1:-3] plot( \x, {\x + 2}) ;
  \draw [domain=1:3] plot( \x, {\x - 2}) ;
  \draw [domain=-6:-2] plot( \x, {\x + 2*\rayonH + 2}) ;
  \draw [domain=2:6] plot( \x, {\x - 2*\rayonH - 2}) ;

  \filldraw [shift={( 1 , {-\rayonH-1} )}] 
    \drawhatim ;
  \filldraw [shift={( -1 , {\rayonH+1} )}] 
    \drawhatim ;
  \filldraw [shift={( {-2*\rayonH-3} , {-\rayonH-1} )}] 
    \drawhatim ;
  \filldraw [shift={( {2*\rayonH+3} , {\rayonH+1} )}] 
    \drawhatim ;

  \filldraw [shift={( {-\rayonH-1} , {-\rayonH-1} )}] 
    \drawsaft ;
  \filldraw [shift={( {\rayonH+1} , {\rayonH+1} )}] 
    \drawsaft ;
  \filldraw [shift={( {-\rayonH-2} , 0 )}, rotate=45] 
    \drawsaft ;
  \filldraw [shift={( {\rayonH+2} , 0 )}, rotate=45] 
    \drawsaft ;
}

\newcommand{\squelettea}{%
  \definelength

  \begin{scope}
    \clip (-4,-4) rectangle (4,4) ;
    \draw plot( \x, {\x + \rayonH}) ;
    \draw plot( \x, {\x - \rayonH}) ;
    \draw plot(-\x, {\x + \rayonH}) ;
    \draw plot(-\x, {\x - \rayonH}) ;
  \end{scope}

  \draw ( {-1-\rayonH}, {-1-\rayonH} ) rectangle
        ( {+1+\rayonH}, {+1+\rayonH} ) ;
  \draw ( {-3-\rayonH}, {-3-\rayonH} ) rectangle
        ( {+3+\rayonH}, {+3+\rayonH} ) ;

  \filldraw [shift={( {+2+\rayonH}, {+2+\rayonH} )}] 
    \drawhatim ;
  \filldraw [shift={( {+2+\rayonH}, {-2-\rayonH} )}] 
    \drawhatim ;
  \filldraw [shift={( {-2-\rayonH}, {+2+\rayonH} )}] 
    \drawhatim ;
  \filldraw [shift={( {-2-\rayonH}, {-2-\rayonH} )}] 
    \drawhatim ;

  \filldraw [shift={( 0, {-2-\rayonH} )}] 
    \drawsaft ;
  \filldraw [shift={( 0, {+2+\rayonH} )}] 
    \drawsaft ;
  \filldraw [shift={( {-2-\rayonH}, 0 )}, rotate=90] 
    \drawsaft ;
  \filldraw [shift={( {+2+\rayonH}, 0 )}, rotate=90] 
    \drawsaft ;
}

\newcommand{\drawfluoX}{%
  ( 0.19,-0.32) .. controls ( 0.22,-0.40) and ( 0.23,-0.43) ..
  ( 0.50,-0.79) .. controls ( 0.60,-0.95) and ( 0.66,-1.12) ..
  ( 0.66,-1.32) .. controls ( 0.65,-1.78) and ( 0.30,-1.98) ..
  ( 0.00,-1.98) .. controls (-0.30,-1.98) and (-0.65,-1.78) ..
  (-0.66,-1.32) .. controls (-0.65,-1.12) and (-0.60,-0.95) ..
  (-0.50,-0.79) .. controls (-0.23,-0.43) and (-0.22,-0.40) ..
  (-0.19,-0.32)
}

\newcommand{\drawRX}{%
  ( 0.15,-0.06) to [out=-87, in=104] 
  ( 0.19,-0.32) .. controls ( 0.21,-0.40) and ( 0.23,-0.46) ..
  ( 0.26,-0.51) .. controls ( 0.31,-0.60) and ( 0.35,-0.65) ..
  ( 0.41,-0.74) .. controls ( 0.57,-0.99) and ( 0.66,-1.30) ..
  ( 0.50,-1.54) .. controls ( 0.39,-1.71) and ( 0.24,-1.81) ..
  ( 0.00,-1.81) .. controls (-0.24,-1.81) and (-0.39,-1.71) ..
  (-0.50,-1.54) .. controls (-0.66,-1.30) and (-0.57,-0.99) ..
  (-0.41,-0.74) .. controls (-0.35,-0.65) and (-0.31,-0.60) ..
  (-0.26,-0.51) .. controls (-0.23,-0.46) and (-0.21,-0.40) ..
  (-0.19,-0.32) to [out=76, in=-93] 
  (-0.15,-0.06)
}

\newcommand{\drawinterne}{%
  ( 0.11,-0.06) .. controls ( 0.13,-0.46) and ( 0.23,-0.53) .. 
  ( 0.34,-0.71) .. controls ( 0.50,-0.96) and ( 0.56,-1.21) .. 
  ( 0.42,-1.43) .. controls ( 0.32,-1.59) and ( 0.19,-1.65) .. 
  ( 0.00,-1.65) .. controls (-0.19,-1.65) and (-0.32,-1.59) .. 
  (-0.42,-1.43) .. controls (-0.56,-1.21) and (-0.50,-0.96) .. 
  (-0.34,-0.71) .. controls (-0.23,-0.53) and (-0.13,-0.46) .. 
  (-0.11,-0.06)
}

\newcommand{\drawaugier}{%
  (-0.15, 0.00) rectangle ( 0.15, -0.06)
}

\newcommand{\MEBpoire}{%
  \begin{scope}[
    scale=2.3,
    every node/.style={font=\scriptsize},
    line cap=round,
  ]

    \pgfmathsetmacro{\xA}{0.145}
    \pgfmathsetmacro{\yA}{-0.35}
    \pgfmathsetmacro{\xB}{0.495}
    \pgfmathsetmacro{\yB}{-1.11}
    \coordinate (A1) at (-\xA, \yA) ;
    \coordinate (A2) at ( \xA, \yA) ;
    \coordinate (B1) at (-\xB, \yB) ;
    \coordinate (B2) at ( \xB, \yB) ;

    % .. Remplissage ..
    % ... Fluo X ...
    \fill [DarkSlateBlue!20!white] \drawfluoX ;
    % ... RX ...
    \fill [DarkSlateBlue!30!white] \drawRX ;
    % ... RX carac ...
    \fill [DarkSlateBlue!40!white] \drawinterne ;

    \begin{scope}
      \clip \drawinterne ;
      % ... Secondaires ...
      \fill [DarkSlateBlue!50!white] 
            (-\xB, \yB) rectangle ( \xB, \yA) ;
      % ... Rétrodiff ...
      \fill [DarkSlateBlue!60!white] 
            (-\xA, \yA) rectangle ( \xA, 0.00) ;
    \end{scope}
    % Électrons Augier
    \fill [fill=DarkSlateBlue!70!white] \drawaugier ;

    % .. Contours ..
    \begin{scope}[DarkSlateBlue]
      % ... Fluo X ...
      \draw \drawfluoX ;
      % ... RX ...
      \draw \drawRX ;
      % ... RX carac ...
      \draw \drawinterne ;
      % Limite secondaires / rétrodiff
      \draw (A1) -- (A2) ;
      % limite rétrodiff / RC carac
      \draw (B1) -- (B2) ;
      % Électrons Augier
      \draw \drawaugier ;
    \end{scope}

    % .. Faisceau incident ..
    \filldraw [draw=DarkRed, bottom color=DarkRed, top color=white]
          (-0.30, 0.77) -- (-0.15, 0.00) -- ( 0.00,  0.00) -- 
          ( 0.15, 0.00) -- ( 0.30, 0.77) 
          node (G) {} ;

    % .. Surface échantillon ..
    \draw [thick]
          (-1.40, 0.00) -- ( 1.40, 0.00) 
          node (H) [pos=1] {} ;

    % .. Légende ..
    \coordinate (A1) at (-0.10,-0.03) ;
    \coordinate (A2) at (-0.50,-0.15) ;
    \coordinate (B1) at (-0.05,-0.20) ;
    \coordinate (B2) at (-0.50,-0.45) ;
    \coordinate (C1) at ( 0.10,-0.80) ;
    \coordinate (C2) at ( 0.80,-0.80) ;
    \coordinate (D1) at ( 0.20,-1.30) ;
    \coordinate (D2) at ( 0.80,-1.30) ;
    \coordinate (E1) at (-0.50,-1.40) ;
    \coordinate (E2) at (-0.80,-1.40) ;
    \coordinate (F1) at (-0.50,-1.65) ;
    \coordinate (F2) at (-0.80,-1.75) ;
  \end{scope}
}

\newcommand{\MEBlegende}{%
  % .. Légende ..
  \begin{scope}[every node/.style={font=\smaller}]
    \draw (A1) -- (A2) 
          node [left] {électrons Auger} ;
    \draw (B1) -- (B2) 
          node [left, align=right] {électrons\\secondaires} ;
    \draw (C1) -- (C2) 
          node [right, align=left] {électrons\\rétrodiffusés} ;
    \draw (D1) -- (D2) 
          node [right, align=left] {\RX\\caractéristiques} ;
    \draw (E1) -- (E2) 
          node [left, align=right] {continuum\\de \RX} ;
    \draw (F1) -- (F2) 
          node [left] {fluorescence~X} ;
    \node [below right, align=left] at (G) 
          {faisceau incident\\d'électrons} ;
    \node [right, align=left] at (H) {surface de\\l'échantillon} ;
  \end{scope}
}

\newcommand{\CLapp}{%
  \begin{scope}[
    scale=0.1, 
    rounded corners=0.5, 
    >=stealth', 
    thick,
  ]
    % \draw [help lines, step=1, ultra thin, densely dotted] 
    %       (-40.00,-10.00) grid ( 40.00, 60.00) ;
    % \draw [help lines, step=5, ultra thin] 
    %       (-40.00,-10.00) grid ( 40.00, 60.00) ;

    \begin{scope}
      % Moniteur
      \draw ( 0.00,  0.00) rectangle ++ ( 30.00, 24.00) ;
      \draw ( 2.00,  8.00) rectangle ++ ( 26.00, 14.00) ;
      \draw ( 2.00,  2.00) rectangle ++ (  2.00,  2.00) ;
      % Système d'acquisition
      \draw ( 0.00, 28.00) rectangle ++ ( 30.00,  6.00) ;
      \draw ( 0.00, 38.00) rectangle ++ ( 30.00,  6.00) ;
      \draw ( 0.00, 48.00) rectangle ++ ( 30.00,  6.00) ;
    \end{scope}

    \begin{scope}[shift={(-15.00, 20.00)}]
      % Loupe bino
      \draw ( 2.00,  0.00) -- ( 5.80,  4.80) -- ( 5.80, 16.20) --
            ( 2.90, 16.20) -- ( 2.90, 21.20) -- ( 3.10, 21.30) -- 
            ( 3.10, 22.90) -- ( 3.70, 23.40) -- ( 3.70, 24.10) -- 
            (-3.70, 24.10) -- (-3.70, 23.40) -- (-3.10, 22.90) -- 
            (-3.10, 21.30) -- (-2.90, 21.20) -- (-2.90, 16.20) -- 
            (-5.80, 16.20) -- (-5.80,  4.80) -- (-2.00,  0.00) -- 
            cycle
         ;
      \draw (-2.00, 0.00) -- (-1.00, 1.00) -- 
            ( 1.00, 1.00) -- ( 2.00, 0.00) ;
      \draw (-2.90, 21.20) -- ( 2.90, 21.20) ;
      \draw (-3.10, 22.90) -- ( 3.10, 22.90) ;
      \draw (-2.90, 18.70) arc (113.4:66.6:7.3) ;
      \draw (-2.90, 18.70) arc (-113.4:-66.6:7.3) ;

      % Caméra CCD
      \draw [shift={( 0.00, 30.00)}, pattern=north west lines]
            ( 4.00,-5.90) -- ( 4.00,-3.50) -- ( 6.00,-3.50) -- 
            ( 6.00, 3.50) -- (-1.60, 3.50) -- (-1.60, 4.80) -- 
            (-3.60, 4.80) -- (-3.60, 3.50) -- (-5.70, 3.50) -- 
            (-7.00, 0.70) -- (-7.00,-0.70) -- (-5.70,-3.50) -- 
            (-4.00,-3.50) -- (-4.00,-5.90) -- cycle
         ;
    \end{scope}

    % Chambre Nuclide
    \begin{scope}[shift={(-27.00, 7.00)}]
      \draw ( 0.00, 0.00) rectangle ++ ( 20.00, 8.00) ;
      \filldraw ( 0.00, 5.00) arc (90:-90:1) ;

      \begin{scope}[shift={( 9.00, 3.25)}]
        \draw ( 0.00, 0.00) rectangle ++ ( 3.00, 1.50) ;
        \draw [->, decorate, decoration={snake, 
               amplitude=.5mm, segment length=2mm, post length=2mm}]
              ( 1.50, 1.5) -- ++ ( 1.50, 8.25) ;
      \end{scope}

      \draw [->] 
            ( 2.00, 4.00) -- ( 7.00, 4.00) 
            node [midway, above] {$e^-$} ;
    \end{scope}

    % Liens
    \draw ( 15.00, 24.00) -- ++ ( 0.00, 4.00) ;
    \draw ( 15.00, 34.00) -- ++ ( 0.00, 4.00) ;
    \draw ( 15.00, 44.00) -- ++ ( 0.00, 4.00) ;
    \draw ( 15.00, 54.00) -- ++ ( 0.00, 4.00) -- ++ 
          (-30.00,  0.00) -- ++ ( 0.00,-4.50) ;

    % % Légende
    \coordinate (A) at (-28.00, 11.00) ;
    \coordinate (B) at (-22.00, 30.00) ;
    \coordinate (C) at (-23.00, 50.00) ;
    \coordinate (D) at ( 31.00, 11.00) ;
    \coordinate (E) at ( 32.00, 54.00) ;
    \coordinate (F) at ( 32.00, 28.00) ;
  \end{scope}
}

\newcommand{\CLappL}{%
  % Légende
  \begin{scope}[every node/.style={font=\smaller}]
    \node [left, align=right] at (A) {chambre\\Nuclide} ;
    \node [left, align=right] at (B) {loupe\\binoculaire} ;
    \node [left, align=right] at (C) {caméra\\3CCD} ;
    \node [right, align=left] at (D) {moniteur} ;

    \draw [decorate, decoration=brace] (E) -- (F)
          node [midway, right, align=left] 
               {système d'acquisition\\et de traitement} ;
  \end{scope}
}

\newcommand{\CLpcp}{%
  \begin{scope}[>=stealth', every node/.style={font=\smaller}]

    % \draw [help lines, densely dotted, red, step=1] 
    %       (-10.00, -20.00) grid ( 60.00, 50.00) ;
    % \draw [help lines, ultra thin, red, step=5] 
    %       (-10.00, -20.00) grid ( 60.00, 50.00) ;

    \begin{scope}[every node/.style={font=\smaller}, darkgray]
      \draw [->] 
            (-0.20,-1.00) -- (-0.20, 5.50) 
            node [below right, align=left] {énergie\\relative} ;
      \node at (-0.80, 4.20) {BC} ;
      \node at (-0.80, 1.75) {BI} ;
      \node at (-0.80,-0.70) {BV} ;
    \end{scope}

    \begin{scope}[thick, pattern=north east lines]
      \filldraw ( 0.50,-0.50) circle (0.20) ;
      \filldraw ( 4.20,-0.50) circle (0.20) ;
    \end{scope}

    \begin{scope}[thick]
      \draw (-0.70, 0.00) -- ( 5.00, 0.00) ;
      \draw (-0.70, 3.50) -- ( 1.20, 3.50) -- 
            ( 1.20, 1.75) -- ( 2.00, 1.75) -- 
            ( 2.00, 3.50) -- ( 5.00, 3.50) ; 
      \draw ( 3.20, 2.50) -- ( 4.20, 2.50) 
            node [right] {\ch{(I^+)^*}} ;
      \draw ( 2.50, 1.00) -- ( 4.20, 1.00) 
            node [right] {\ch{(I^+)}} ;
    \end{scope}

    \begin{scope}[->, densely dotted, thick, DarkSlateBlue]
      \draw ( 0.50,-0.30) -- ( 0.50, 3.80) -- 
            ( 2.70, 3.80) node [midway, above] {$a$} --
            ( 2.70, 1.00) node [pos=0.33, left]  {$b$} ;
      \draw ( 3.40, 1.00) -- ( 3.40, 2.50) 
            node [pos=0.33, left] {$c$} ;
      \draw ( 4.00, 2.50) -- ( 4.00, 1.00) 
            node [pos=0.67, right] {$d$} ;
      \draw ( 4.00, 1.00) -- ( 4.20,-0.30) 
            node [pos=0.45, left] {$e$} ;
    \end{scope}

    \draw [
      shift={( 4.20, 2.00)},
      DarkRed, ->, 
      decorate, decoration={snake, amplitude=.5mm, segment length=2mm,
                            pre length=0.5mm, post length=0.5mm}
    ]     ( 0.00, 0.00) -- ( 1.00,  0.00)
          node [right] {$h\nu$} ;

    \draw [DarkRed, ->, ultra thick]
          ( 1.00,-1.50) -- ( 0.60,-0.70) 
          node [pos=0, right, align=left] 
               {bombardement\\éctronique} ;

    \coordinate (L) at (6.50,-1.00) ;

    \node at (L) [above right, align=left, text width=6.5cm] 
    {
      \begin{itemize}
        \item [BC :] bande de conduction ;
        \item [BI :] bande interdite ;
        \item [BV :] bande de valence.
      \end{itemize}

      \bigskip

      \begin{itemize}
        \item [$a$ :] ionisation d'un atome sans intervention 
              d'un centre piège ;
        \item [$b$ :] capture de l'électron par l'impureté 
              \ch{(I^+)} ;
        \item [$c$ :] excitation en \ch{(I^+)^*} ;
        \item [$d$ :] désexcitation radiative ;
        \item [$e$ :] recapture de l'électron dans la bande 
              de valence (il n'y a pas forcément ionisation, 
              l'électron peut directement être capturé par 
              l'impureté).
      \end{itemize}
    } ;

  \end{scope}
}

\newcommand{\CLpcpL}{%
  \begin{scope}[every node/.style={font=\smaller}]
    \node at (L) [above right, align=left, text width=5cm] 
    {
      \begin{itemize}
        \item [BC :] bande de conduction ;
        \item [BI :] bande interdite ;
        \item [BV :] bande de valence.
      \end{itemize}

      \bigskip

      \begin{itemize}
        \item [$a$ :] ionisation d'un atome sans intervention 
              d'un centre piège ;
        \item [$b$ :] capture de l'électron par l'impureté 
              \ch{(I^+)} ;
        \item [$c$ :] excitation en \ch{(I^+)^*} ;
        \item [$d$ :] désexcitation radiative ;
        \item [$e$ :] recapture de l'électron dans la bande 
              de valence (il n'y a pas forcément ionisation, 
              l'électron peut directement être capturé par 
              l'impureté).
      \end{itemize}
    } ;
  \end{scope}
}

\newcommand{\DXapp}{%
  \begin{scope}[
    % handle active characters in code,
    >=stealth',
    every node/.style={font=\small},
    line cap=round
  ]
    \pgfmathsetmacro{\r}{3.00}
    \pgfmathsetmacro{\a}{20.00}

    \coordinate (O) at ( 0.00, 0.00) ;

    % Cercle gonio
    \draw (O) circle (\r) ;

    % Légende I
    \begin{scope}[DarkSlateBlue]
      \draw [->, shift={(90-\a:0.15*\r)}, rotate=-\a]
            (-0.1*\r, 0.00) arc (170:10:0.1*\r) 
            node [midway, right] {$\uline{\omega}$} ;
      \draw [->] (O) (-2*\a:1.30*\r) arc (-2*\a:-3*\a:1.30*\r) 
            node [above] {$\uline{2\omega}$} ;

      \draw (-0.27*\r, 0.00) arc (180:180-\a:0.27*\r) 
            node [midway, left] {$\theta$} ;
      \draw [rotate=-\a] 
            ( 0.27*\r, 0.00) arc (0:-\a:0.27*\r) ;
      \draw ( 0.47*\r, 0.00) arc (0:-2*\a:0.47*\r) 
            node [pos=0.25, right] {$\theta$} 
            node [pos=0.80, right] {$\theta$} ;
      \draw ( 0.76*\r, 0.00) arc (0:-2*\a:0.76*\r) 
            node [midway, right] {$2\theta$} ;
    \end{scope}

    % Trajet linéaire
    \draw [dashed] (-\r, 0.00) -- ( \r, 0.00) ;

    % Échantillon
    \begin{scope}[rotate=-\a]
      \draw (-0.50*\r, 0.00) -- ( 0.50*\r, 0.00) ;
      \draw [yshift=0.4pt, thick, pattern=north east lines, 
             preaction={fill=white}]
            ( 0.33*\r, 0.00) rectangle (-0.33*\r, 0.10*\r) 
            node (E) [pos=0.85] {} ;
    \end{scope}

    % Trajet rayons
    \begin{scope}[thick, DarkRed, line cap=round]
      \draw [postaction={decorate}, decoration={markings, 
             mark=at position 0.15 with {\arrow[scale=1.4]{>};}}]
            (-1.3*\r, 0.00) -- (O) 
            node [midway, below] {RX incident} ;
      \draw [postaction={decorate}, decoration={markings, 
             mark=at position 0.90 with {\arrow[scale=1.4]{>};}}]
            (O) -- (-2*\a:\r) 
            node [pos=0.67, left, align=left] {RX\\diffracté} ;
    \end{scope}

    % Détecteur
    \begin{scope} [
      thick,
      rounded corners=1,
      shift={(-2*\a:\r)},
      rotate=-90-2*\a,
      fill=white,
    ]
      \filldraw (-0.09*\r, 0.00) rectangle ( 0.09*\r, 0.10*\r) ;
      \filldraw (-0.12*\r, 0.10*\r) rectangle ( 0.12*\r, 0.40*\r) 
            node (D) [pos=0.45] {} ;
    \end{scope}

    % Fente
    \begin{scope}
      \shorthandoff{;}
      \tikzmath{
        \e = 0.03*\r ;          % épaisseur
        \l = 0.19*\a ;          % largeur angulaire
        \h = \r ;               % rayon de courbure
        \y = \h * sin(\l) ;
        \x = \h * (1-cos(\l)) ;
        \b = 4.90 ;
      }
      \shorthandon{;}
      \foreach \s in {-1, 1} {%
        \filldraw [rotate=-2*\a+\s*\b, shift={( \r, 0.00)}]
              (-\x, -\y)  arc (-\l: \l:\h) -- ++ 
              ( \e, 0.00) node (F) [pos=0.5] {} 
               arc ( \l:-\l:\h) -- cycle ;
      }
    \end{scope}

    % Ampli / enregistreur
    \begin{scope}[shift={( 6.00, 1.00)}, rounded corners=1]
      \begin{scope}[thick]
        \draw ( 0.00, 0.00) rectangle ++ ( 6.00,-4.00) 
              node (G) [midway, shift={( 0.00, 1.50)}] {} ;
        \draw ( 0.00, 0.00) rectangle ++ ( 6.00, 1.40) 
              node (A) [midway] {} ;
      \end{scope}
      \begin{scope}[->]
        \draw ( 0.50,-3.50) -- ++ ( 5.00, 0.00) 
              node [below] {\scriptsize$2\theta$} ;
        \draw ( 0.50,-3.50) -- ++ ( 0.00, 2.50) 
              node [above] {\scriptsize$\uline{T}$} ;
      \end{scope}

      % Diffractogramme
      \begin{scope}[shift={( 0.50,-3.30)}, x=0.55mm, y=0.4mm]
        \draw (  0.06,  0.63) .. 
              controls (  4.85,  0.20) and (  3.95,  6.14) .. 
              (  4.35,  9.27) -- (  5.13, 36.84) --
              (  6.20,  2.81) .. 
              controls (  7.37, -1.02) and ( 12.72,  0.93) .. 
              ( 15.76,  0.78) .. 
              controls ( 19.59,  1.22) and ( 19.37,  5.78) .. 
              ( 19.79,  8.65) -- ( 20.99, 32.80) --
              ( 22.78,  5.59) .. 
              controls ( 23.20,  0.50) and ( 28.08,  0.44) .. 
              ( 31.76,  0.41) .. 
              controls ( 39.54,  0.61) and ( 47.31,  0.40) .. 
              ( 55.09,  0.37) .. 
              controls ( 58.72,  0.50) and ( 61.02,  0.87) .. 
              ( 62.55,  5.38) -- ( 63.48, 45.63) --
              ( 63.61, 53.65) -- ( 64.34, 20.19) ..
              controls ( 64.60, 13.97) and ( 64.80,  7.61) .. 
              ( 65.40,  1.52) .. 
              controls ( 68.18, -0.46) and ( 71.96,  0.68) .. 
              ( 75.15,  0.64) .. 
              controls ( 79.53,  0.10) and ( 78.66,  5.75) .. 
              ( 79.21,  8.77) -- ( 79.65, 22.73) --
              ( 80.35, 11.69) .. 
              controls ( 80.50, 10.18) and ( 81.02,  6.56) .. 
              ( 81.36,  5.14) -- ( 82.98, 23.23) --
              ( 83.55, 23.99) -- ( 84.85,  5.20) ..
              controls ( 85.35,  1.09) and ( 87.93,  1.12) .. 
              ( 91.11,  0.00) ;

        \node (P1) at (  5.13, 3.00) {} ;
        \node (P2) at ( 20.99, 3.00) {} ;
      \end{scope}
    \end{scope}

    % Liens
    \draw ( 0.00, 0.00) (-2*\a:1.4*\r) -- ++ 
          ( 2.30, 0.00) -- ++ ( 0.00, 4.40) -- ( 6.00, 1.70) ;

    % Légende II
    \begin{scope}[Gray!25!DarkGray]
      \shorthandoff{;}
      \tikzmath{
        \x = \r*cos(45) ;
        \y = \r*sin(45) ;
      }
      \shorthandon{;}
      \node at (\x, \y) 
            [right, shift={( 0.20, 0.00)}, align=left] 
            {cercle\\goniométrique} ;

      \node at (A) {Ampli} ;
      \node at (G) {Enregistreur} ;

      \draw (E.center) -- ++ (-0.08*\r, 0.23*\r) 
            node [above] {échantillon} ;

      \draw (D.center) -- ++ ( 0.20*\r, 0.10*\r) 
            node [right] {détecteur} ;

      \draw (F.center) -- ++ ( 0.20*\r, 0.10*\r) 
            node [right] {fente} ;

      \draw (P1.center) -- ++ ( 0.05*\r, -0.32*\r) -- (P2.center)
            node [pos=0, below] {pics de diffraction} ;
    \end{scope}
  \end{scope}
}

\newcommand{\DXpcp}{%
  \tikzfading[
    name=fadeout,
    inner color=transparent!0,
    outer color=transparent!100,
  ]

  \tikzset{
    onde/.style={
      thick,
      line cap=round,
      decorate,
      decoration={
        snake,
        mirror,
        amplitude=.7mm,
        segment length=2.5mm,
        % segment length=5mm,
        % pre length=0.31mm,
        post length=1.5mm,
      }
    }
  }

  \begin{scope}[
    >=stealth', 
    line cap=round, 
  ]

    \coordinate (O) at ( 0.00, 0.00) ;
    \pgfmathsetmacro{\xm}{4.00}    % x min/max
    \pgfmathsetmacro{\t}{30.00}    % theta
    \pgfmathsetmacro{\d}{1.00}     % d

    \foreach \y in {0.00, 1.00, 2.00} {
      \draw (-\xm, \y) -- ( \xm, \y) ;
      \foreach \x in {-3.00, -1.50, ..., 4.00} {
        \fill [DarkSlateBlue, path fading=fadeout] 
              (\x, \y) circle (0.2) ;
      }
    }

    \shorthandoff{;}
    \tikzmath{
      \x = \xm ;
      \y = \x * tan(\t) ;
      \l = 0.25 ;     % longueur d'onde
      % \l = \x / cos(\t) ;
      \b = \d * cos(\t) ;
    }
    \shorthandon{;}

    \foreach \o in {0.00, 1.00} {
      \begin{scope}[every path/.style={onde}]
        \draw [->] 
              ( 0.00, \o)  -- ( \x, \y+\o) ;
        \draw [-<, decoration={mirror=false}] 
              ( 0.00, \o) -- (-\x, \y+\o) ;
      \end{scope}

      \draw [very thin] 
            (-\x, \y+\o) -- ( 0.00, \o) -- ( \x, \y+\o) ;
    }

    \begin{scope}[DarkRed]
      \draw ( 0.00, \d) -- (O) ;
      \foreach \s in {-1, 1} {
        \foreach \o in {0, 4.75} {
          \draw [xscale=\s]
                ( 0.00, \d) ++ (\t:0.015+\o*\l)     -- ++ (-90+\t:\b) ;
        }
        \draw [xscale=\s, very thin]
              ( 0.50, \d) arc (0:\t:0.5) ;
        \node at ([xscale=\s] 0.60, 1.12) {\smaller$\theta$} ;
        \draw [xscale=\s, very thin]
              ( 0.00, 0.50) arc (-90:-89+\t:0.5) ;
        \node at ([xscale=\s] 0.18, 0.40) {\smaller$\theta$} ;
      }

      \draw [|-|, shift={(180-\t:0.015+12.75*\l)}]
        ([shift={(90-\t:0.15)}] 0.00, \d) -- ++ (180-\t:1*\l) 
        node [pos=0, above] {\smaller$\lambda$} ;

      \draw [<->, shorten >=1pt, shorten <=1pt, ] 
            (\xm-0.2, 0.00) -- ++ ( 0.00, \d) 
            node [midway, right] {\smaller$d$} ;
    \end{scope}

  \end{scope}
}


% Triangle minimal
\newcommand{\triangleminimal}{%
  \tikzset{%
    add/.style args={##1 and ##2}{
      to path={%
       ($(\tikztostart)!-##1!(\tikztotarget)$) -- %
       ($(\tikztotarget)!-##2!(\tikztostart)$)\tikztonodes%
      },%
      add/.default={.2 and .2}%
    },
    forme/.style={
      draw, blue, fill=blue!25
    },
    ligne/.style={
      red
    },
    fleche/.style={
      thin,
      ->,
      >=stealth,
      shorten >=1pt
    }
  }

  \definelength
  \pgfmathsetmacro{\angle}{22.5} ;
  % \pgfmathsetmacro{\rayonH}{sqrt(2)} ;
  % \pgfmathsetmacro{\lenH}{(sqrt(2)-1)/cos(45)} ;
  \pgfmathsetmacro{\rayon}{(1+(3+\rayonH)/tan(\angle))} ;

  \coordinate (O) at (0, 0) ;
  \coordinate (O1) at ([shift={(180:\rayon)}] O) ;
  \coordinate (O2) at ([shift={(135:\rayon)}] O) ;

  \clip ({-(0.5+\rayon+\rayonH)}, {-(0.5+\rayonH)}) rectangle 
        ({0.5+0}, {0.5+\rayonH+cos(45)*\rayon}) ;

  \draw [name path=cercleO, gray, thin, dashed] 
        (O) circle (\rayon) ;

  \draw [name path=lgnOO1] (O) -- (180:\rayon+2) ;
  \draw [name path=lgnOO2] (O) -- (135:\rayon+2) ;
  \draw [name path=lgnOH, dashed] (O) -- (157.5:\rayon+3) ;

  \node [right] at (O) {} ;
  \node [above left] at (O1) {} ;
  \node [left] at (O2) {} ;

  \begin{scope}[gray, thin, dashed]
    \draw (O) circle (\rayonH) ;
    \draw ([shift={(-1,-1)}] O) rectangle ([shift={(1,1)}] O) ;
    \draw [rotate=45] ([shift={(-1,-1)}] O) rectangle ([shift={(1,1)}] O) ;
    \draw (O1) circle (\rayonH) ;
    \draw ([shift={(-1,-1)}] O1) rectangle ([shift={(1,1)}] O1) ;
    \draw [rotate=45] ([shift={(-1,-1)}] O1) rectangle ([shift={(1,1)}] O1) ;
    \draw (O2) circle (\rayonH) ;
    \draw ([shift={(-1,-1)}] O2) rectangle ([shift={(1,1)}] O2) ;
    \draw [rotate=45] ([shift={(-1,-1)}] O2) rectangle ([shift={(1,1)}] O2) ;
  \end{scope}

  \draw [dotted, name path=lgn1] 
        ([shift={(45:\rayonH)}] O1) --++ (45:9) ;
  \draw [dotted, name path=lgn2] 
        ([shift={(-90:\rayonH)}] O2) --++ (-90:9) ;
  \node [name intersections={of=lgn1 and lgn2}] 
        (H1) at (intersection-1) {} ;

  \draw [dotted, name path=lgn1] 
        ([shift={(90:\rayonH)}] O1) --++ (90:4) ;
  \draw [dotted, name path=lgn2] 
        ([shift={(225:\rayonH)}] O2) --++ (225:4) ;
  \node [name intersections={of=lgn1 and lgn2}] 
        (H2) at (intersection-1) {} ;

  % Forme 1 - Octogone
  % ===================================================================

  \draw [dotted, name path=lgnO1ne90] 
        (O1) ++ (45:\rayonH) --++ (90:4) ;
  \draw [dotted, name path=lgnO2s225] 
        (O2) ++ (-90:\rayonH) --++ (225:4) ;

  \draw [dotted, name path=lgnO1nw90] 
        (O1) ++ (135:\rayonH) --++ (90:5) ;
  \draw [dotted, name path=lgnO2w225] 
        (O2) ++ (180:\rayonH) --++ (225:5) ;

  \draw [dotted, name path=lgnO1n135] 
        (O1) ++ (90:\rayonH) --++ (135:2) ;
  \draw [dotted, name path=lgnO1n45] 
        (O1) ++ (90:\rayonH) --++ (45:9) ;

  \draw [dotted, name path=lgnO2sw180] 
        (O2) ++ (225:\rayonH) --++ (180:2) ;
  \draw [dotted, name path=lgnO2sw270] 
        (O2) ++ (225:\rayonH) --++ (270:9) ;

  \path (O1) ++ (90:\rayonH) node (Aa) {} ;
  \path (O2) ++ (225:\rayonH) node (Ae) {} ;
  \node [name intersections={of=lgnO1n45 and lgnO1ne90}] 
        (Ab) at (intersection-1) {} ;
  \node [name intersections={of=lgnO2s225 and lgnO1ne90}] 
        (Ac) at (intersection-1) {} ;
  \node [name intersections={of=lgnO2s225 and lgnO2sw270}] 
        (Ad) at (intersection-1) {} ;
  \node [name intersections={of=lgnO2sw180 and lgnO2w225}] 
        (Af) at (intersection-1) {} ;
  \node [name intersections={of=lgnO1nw90 and lgnO2w225}] 
        (Ag) at (intersection-1) {} ;
  \node [name intersections={of=lgnO1n135 and lgnO1nw90}] 
        (Ah) at (intersection-1) {} ;

  % 
  % ===================================================================
  \draw [dotted, name path=lgnO1On] 
        ([shift={(45:\rayonH)}] O1) -- ([shift={(135:\rayonH)}] O) ;
  \draw [dotted, name path=lgnO1Os] 
        ([shift={(-45:\rayonH)}] O1) -- ([shift={(225:\rayonH)}] O) ;
  \draw [dotted, name path=lgnO2Osw] 
        ([shift={(-90:\rayonH)}] O2) -- ([shift={(180:\rayonH)}] O) ;
  \draw [dotted, name path=lgnO2Oen] 
        ([shift={(0:\rayonH)}] O2) -- ([shift={(90:\rayonH)}] O) ;

  % 
  % ===================================================================
  \draw [dotted, name path=lgnO1e45] 
        ([shift={(0:\rayonH)}] O1) --++ (45:6) ;
  \draw [dotted, name path=lgnO2se270] 
        ([shift={(-45:\rayonH)}] O2) --++ (-90:6) ;

  \draw [dotted, name path=lgnOwsw157] 
        ([shift={(202.5:\rayonH)}] O) --++ (157.5:10) ;
  \draw [dotted, name path=lgnOnnw157] 
        ([shift={(112.5:\rayonH)}] O) --++ (157.5:10) ;

  \node [name intersections={of=lgnO1e45 and lgnO2se270}] 
        (Ba) at (intersection-1) {} ;
  \node [name intersections={of=lgnO1e45 and lgnOnnw157}] 
        (Bb) at (intersection-1) {} ;
  \node [name intersections={of=lgnO2Osw and lgnOnnw157}] 
        (Bc) at (intersection-1) {} ;
  \node [name intersections={of=lgnO1On and lgnO2Osw}] 
        (Bd) at (intersection-1) {} ;
  \node [name intersections={of=lgnO1On and lgnOwsw157}] 
        (Be) at (intersection-1) {} ;
  \node [name intersections={of=lgnOwsw157 and lgnO2se270}] 
        (Bf) at (intersection-1) {} ;

  \draw [dotted, name path=lgnOwnw202] 
        ([shift={(157.5:\rayonH)}] O) --++ (202.5:5) ;
  \draw [dotted, name path=lgnOwnw112] 
        ([shift={(157.5:\rayonH)}] O) --++ (112.5:5) ;

  \node [draw, gray, thin, dashed, name path=cercleBa] 
        at (O) [circle through={(Ba)}] {};
  \node [draw, gray, thin, dashed, name path=cercleBb] 
        at (O) [circle through={(Bb)}] {};
  \node [draw, gray, thin, dashed, name path=cercleBd] 
        at (O) [circle through={(Bd)}] {};

  \node [name intersections={of=lgnOO1 and cercleBa}] 
        (B1a) at (intersection-1) {} ;
  \node [name intersections={of=lgnOO2 and cercleBa}] 
        (B2a) at (intersection-1) {} ;

  \node [name intersections={of=lgnOO1 and cercleBd}] 
        (B1d) at (intersection-1) {} ;
  \node [name intersections={of=lgnOO2 and cercleBd}] 
        (B2d) at (intersection-1) {} ;

  \node [name intersections={of=lgnO1On and cercleBb}] 
        (B1f) at (intersection-1) {} ;
  \node [name intersections={of=lgnO1Os and cercleBb}] 
        (B1b) at (intersection-1) {} ;

  \node [name intersections={of=lgnO2Osw and cercleBb}] 
        (B2b) at (intersection-1) {} ;
  \node [name intersections={of=lgnO2Oen and cercleBb}] 
        (B2f) at (intersection-1) {} ;

  \node [name intersections={of=lgnOwnw112 and lgnO2Oen}] 
        (B2e) at (intersection-1) {} ;
  \node [name intersections={of=lgnOwnw202 and lgnO1Os}] 
        (B1c) at (intersection-1) {} ;

  \draw [name path=lgnB1aB2a, add= 0.3 and 0.3, dotted] 
        (B1a) to (B2a) ;
  \draw [name path=lgnB2aB2f, add= 0.3 and 4.5, dotted] 
        (B2f) to (B2a) ;
  \draw [name path=lgnB1aB1b, add= 0.3 and 4.5, dotted] 
        (B1b) to (B1a) ;

  \node [name intersections={of=lgnO2se270 and lgnOnnw157}] 
        (Cb) at (intersection-1) {} ;
  \node [name intersections={of=lgnB2aB2f and lgnOnnw157}] 
        (Cc) at (intersection-1) {} ;
  \node [name intersections={of=lgnB2aB2f and lgnB1aB1b}] 
        (Cd) at (intersection-1) {} ;
  \node [name intersections={of=lgnB1aB1b and lgnOwsw157}] 
        (Ce) at (intersection-1) {} ;
  \node [name intersections={of=lgnO1e45 and lgnOwsw157}] 
        (Cf) at (intersection-1) {} ;

  \node [name intersections={of=lgnB1aB2a and lgnO1Os}] 
        (C1b) at (intersection-1) {} ;
  \node [name intersections={of=lgnO2sw270 and lgnO1Os}] 
        (C1c) at (intersection-1) {} ;
  \node (C1d) at ([shift={(0:\rayonH)}] O1) {} ;
  \node [name intersections={of=lgnO2sw270 and lgnO1On}] 
        (C1e) at (intersection-1) {} ;
  \node [name intersections={of=lgnB1aB1b and lgnO1On}] 
        (C1f) at (intersection-1) {} ;

  \node [name intersections={of=lgnB2aB2f and lgnO2Osw}] 
        (C2b) at (intersection-1) {} ;
  \node (C2d) at ([shift={(-45:\rayonH)}] O2) {} ;
  \node [name intersections={of=lgnB1aB2a and lgnO2Oen}] 
        (C2f) at (intersection-1) {} ;

  \node [name intersections={of=lgnO1n45 and lgnO2Osw}] 
        (C2c) at (intersection-1) {} ;
  \node [name intersections={of=lgnO1n45 and lgnO2Oen}] 
        (C2e) at (intersection-1) {} ;

  \node (D1a) at ([shift={(45:\rayonH)}] O1) {} ;
  \node (D2a) at ([shift={(-90:\rayonH)}] O2) {} ;
  \node [name intersections={of=lgnO1n45 and lgnOwsw157}] 
        (D1b) at (intersection-1) {} ;
  \node [name intersections={of=lgnO2sw270 and lgnOnnw157}] 
        (D2b) at (intersection-1) {} ;
  \node [name intersections={of=lgnO1e45 and lgnB1aB1b}] 
        (D1c) at (intersection-1) {} ;
  \node [name intersections={of=lgnO2se270 and lgnB2aB2f}] 
        (D2c) at (intersection-1) {} ;

  % Fond blanc
  % ===================================================================
  \fill [white]
        ([shift={(1,-2)}] O) rectangle ({-(\rayon+3)}, {\rayon-1}) ;
  \draw ({-(0.5+\rayon+\rayonH)}, {-(0.5+\rayonH)}) rectangle
        ({0.5+0}, {0.5+\rayonH+cos(45)*\rayon}) ;

  % Hatims
  % ===================================================================
  % \draw [red, thick] (O) \drawhatim ;
  \draw [forme] (O1) \drawhatim ;
  \draw [forme] (O2) \drawhatim ;

  % Octogone
  % ===================================================================
  \draw [forme] 
           (Aa.center)
        -- (Ab.center)
        -- (Ac.center)
        -- (Ad.center)
        -- (Ae.center)
        -- (Af.center)
        -- (Ag.center)
        -- (Ah.center)
        -- cycle
        ;

  % Hexagones allongés
  % ===================================================================
  % Central
  % -------------------------------------------------------------------
  \draw [forme] 
           (Ba.center)
        -- (Bb.center)
        -- (Bc.center)
        -- (Bd.center)
        -- (Be.center)
        -- (Bf.center)
        -- cycle
        ;
  % Bas
  % -------------------------------------------------------------------
  \draw [forme] 
           (B1a.center)
        -- (B1b.center)
        -- (B1c.center)
        -- (B1d.center)
        -- (Be.center)
        -- (B1f.center)
        -- cycle
        ;
  % Haut
  % -------------------------------------------------------------------
  \draw [forme] 
           (B2a.center)
        -- (B2b.center)
        -- (Bc.center)
        -- (B2d.center)
        -- (B2e.center)
        -- (B2f.center)
        -- cycle
        ;

  % Hexagones
  % ===================================================================
  % Central
  % -------------------------------------------------------------------
  \draw [forme] 
           (Ba.center)
        -- (Cb.center)
        -- (Cc.center)
        -- (Cd.center)
        -- (Ce.center)
        -- (Cf.center)
        -- cycle
        ;
  % Bas
  % -------------------------------------------------------------------
  \draw [forme] 
           (B1a.center)
        -- (C1b.center)
        -- (C1c.center)
        -- (C1d.center)
        -- (C1e.center)
        -- (C1f.center)
        -- cycle
        ;
  % Haut
  % -------------------------------------------------------------------
  \draw [forme] 
           (B2a.center)
        -- (C2b.center)
        -- (C2c.center)
        -- (C2d.center)
        -- (C2e.center)
        -- (C2f.center)
        -- cycle
        ;

  % Hexagones rotation
  % ===================================================================
  % Bas
  % -------------------------------------------------------------------
  \draw [forme] 
           (D1a.center)
        -- (Ab.center)
        -- (D1b.center)
        -- (Ce.center)
        -- (D1c.center)
        -- (C1e.center)
        -- cycle
        ;
  % Haut
  % -------------------------------------------------------------------
  \draw [forme] 
           (D2a.center)
        -- (Ad.center)
        -- (D2b.center)
        -- (Cc.center)
        -- (D2c.center)
        -- (C2c.center)
        -- cycle
        ;

  \draw (O) -- (O1) ;
  \draw (O) -- (O2) ;
  \draw (O1) arc (180:110:\rayon) ;
  \draw (O1) arc (180:200:\rayon) ;
  \draw [densely dotted] (O1) --++ (45:5.5) ;
  \draw [densely dotted] (O2) --++ (-90:5.5) ;


  \begin{scope}[red, densely dashed]
    \draw (O) ++ (-22.5:2) --++ (157.5:20) ;
    \draw (O) -- (O1) arc (180:135:\rayon) -- cycle ;
  \end{scope}
  % \draw [ligne] (Cc) circle (0.3) ;

  \begin{scope}[red, font=\scriptsize]
    \node [text width=1cm] 
          (X) at (-\rayon, 8.5) {axe de symétrie} ;
    \node [text width=1cm, align=right] 
          (T) at (-1.5, 5.5) {triangle minimal} ;

    \draw [fleche, out=-90, in=45] 
          (T) to ($(O)!0.1!(O2)$) ;
    \draw [fleche, out=-90, in=60] 
          (X) to ($(O)!1.12!(H2)$) ;
  \end{scope}
}


% =====================================================================
% =====================================================================
% == Système Lab                                                     ==
% =====================================================================
% =====================================================================
\newcommand{\cylindre}[1]{%
  \pgfmathsetmacro{\color}{-10*#1+50}
  \draw [fill=black!\color]
        ({-\Rx}, {\Rz/2+#1*(\of+\Rz)}) arc (180:360:{\Rx} and {\Ry}) 
        -- ++ (-90:\Rz) 
        arc (360:180:{\Rx} and {\Ry}) 
        -- cycle
        ;
  \draw [fill=white]
        (0, {\Rz/2+#1*(\of+\Rz)}) ellipse ({\Rx} and {\Ry}) ;
}

\newcommand{\secteur}[4]{%
  \draw [shading=#3, shading angle=#4] 
        ({\Rxp*cos(#1)}, {\Ryp*sin(#1)}) 
        arc ({#1}:{#2}:{\Rxp} and {\Ryp}) 
        -- ({\Rxg*cos(#2)}, {\Ryg*sin(#2)})
        arc ({#2}:{#1}:{\Rxg} and {\Ryg}) 
        -- cycle
        ;
}

\newcommand{\cadre}[4]{%
  % Angle de l'équation de l'ellipse
  \pgfmathsetmacro{\ang}{#1}
  % Angle géométrique
  \pgfmathsetmacro{\angb}{#2}
  % Numéro du cadre
  \pgfmathsetmacro{\num}{#3}
  % Coordonnées du point au bord du cylindre
  \pgfmathsetmacro{\xa}{\Rx*cos(\ang)}
  \pgfmathsetmacro{\ya}{\Ry*sin(\ang)}
  % longeur (abscisses) d'un cadre = 1/3 de la distance cylindre/roue
  \pgfmathsetmacro{\len}{(\Rxp-\Rx)*cos(\ang)/3}
  \pgfmathsetmacro{\offset}{0.1*(sign(90-abs(\angb)))}

  % Coordonnées du centre du cadre
  \pgfmathsetmacro{\xc}{\xa+((\num-1+0.5)*\len)}
  \pgfmathsetmacro{\yc}{\xc*tan(\angb)}
  % Coordonnées de l'angle haut gauche
  \pgfmathsetmacro{\xb}{\xc-(0.5*(\len-\offset))}
  \pgfmathsetmacro{\yb}{\yc+(0.5*\Rz-0.5*(\len-\offset)*tan(\angb))}

  \draw [very thin, fill=#4]
        (\xb, \yb)
        -- ++ (\angb:{(\len-\offset)/cos(\angb)})
        -- ++ (-90:\Rz)
        -- ++ ({180+\angb}:{(\len-\offset)/cos(\angb)})
        -- cycle
        ;
}

\newcommand{\rangee}[3]{%
  % Angle de l'équation de l'ellipse
  \pgfmathsetmacro{\ang}{#1}
  % Angle géométrique
  \pgfmathsetmacro{\angb}{#2}
  % Nombre de cadres dans la rangée
  \pgfmathsetmacro{\num}{#3}
  % Coordonnées de l'angle supérieur côté cylindre
  \pgfmathsetmacro{\xa}{\Rx*cos(\ang)}
  \pgfmathsetmacro{\ya}{\Ry*sin(\ang)+0.5*\Rz}
  % \node at (\xa, \ya) {$\times$} ;
  % longeur (abscisses) d'un cadre = 1/3 de la distance cylindre/roue
  \pgfmathsetmacro{\len}{\num*(\Rxp-\Rx)*cos(\ang)/3}

  \fill [very thin, yellow!5, draw]
        (\xa, \ya)
        -- ++ (\angb:{\len/cos(\angb)})
        -- ++ (-90:\Rz)
        -- ++ ({180+\angb}:{\len/cos(\angb)})
        -- cycle
        ;
}

\pgfmathsetmacro{\coeff}{2.5}
\pgfmathsetmacro{\Rx}{1.00}
\pgfmathsetmacro{\Ry}{\Rx/\coeff}
\pgfmathsetmacro{\Rz}{2.00}
\pgfmathsetmacro{\of}{0.25}
\pgfmathsetmacro{\Rxp}{7*\Rx}
\pgfmathsetmacro{\Ryp}{\Rxp/\coeff}
\pgfmathsetmacro{\Rxg}{9*\Rx}
\pgfmathsetmacro{\Ryg}{\Rxg/\coeff}

\definecolor{rayon01}{RGB}{ 85,  16,  18}
\definecolor{rayon02}{RGB}{ 91,  16,  31}
\definecolor{rayon03}{RGB}{ 70,  19,  32}
\definecolor{rayon04}{RGB}{ 27,  16,  54}
\definecolor{rayon05}{RGB}{ 22,  33,  76}
\definecolor{rayon06}{RGB}{ 18,  39,  65}
\definecolor{rayon07}{RGB}{ 22,  52,  31}
\definecolor{rayon08}{RGB}{ 99, 110,  36}
\definecolor{rayon09}{RGB}{140, 107,  32}
\definecolor{rayon10}{RGB}{144,  77,  25}

\definecolor{cadre01a}{RGB}{ 50,  13,  13}
\definecolor{cadre01b}{RGB}{ 63,  13,  14}
\definecolor{cadre01c}{RGB}{ 73,  14,  15}
\definecolor{cadre01d}{RGB}{ 87,  15,  18}
\definecolor{cadre01e}{RGB}{ 98,  17,  19}
\definecolor{cadre01f}{RGB}{112,  19,  20}
\definecolor{cadre01g}{RGB}{115,  19,  21}

\definecolor{cadre02a}{RGB}{ 55,  12,  25}
\definecolor{cadre02b}{RGB}{ 67,  14,  27}
\definecolor{cadre02c}{RGB}{ 77,  14,  27}
\definecolor{cadre02d}{RGB}{ 92,  16,  31}
\definecolor{cadre02e}{RGB}{109,  19,  40}
\definecolor{cadre02f}{RGB}{123,  19,  42}

\definecolor{cadre03b}{RGB}{ 51,  16,  27}
\definecolor{cadre03c}{RGB}{ 55,  17,  27}
\definecolor{cadre03d}{RGB}{ 70,  19,  31}
\definecolor{cadre03e}{RGB}{ 81,  21,  36}

\definecolor{cadre04b}{RGB}{ 22,  15,  38}
\definecolor{cadre04c}{RGB}{ 25,  16,  47}
\definecolor{cadre04d}{RGB}{ 28,  18,  56}
\definecolor{cadre04e}{RGB}{ 32,  18,  56}
\definecolor{cadre04f}{RGB}{ 31,  19,  64}

\definecolor{cadre05a}{RGB}{ 20,  21,  41}
\definecolor{cadre05b}{RGB}{ 20,  25,  50}
\definecolor{cadre05c}{RGB}{ 22,  28,  60}
\definecolor{cadre05d}{RGB}{ 24,  33,  74}

\definecolor{cadre06a}{RGB}{ 16,  26,  43}
\definecolor{cadre06b}{RGB}{ 19,  32,  46}
\definecolor{cadre06c}{RGB}{ 18,  34,  57}
\definecolor{cadre06d}{RGB}{ 20,  38,  61}

\definecolor{cadre07a}{RGB}{ 20,  33,  22}
\definecolor{cadre07b}{RGB}{ 22,  38,  25}
\definecolor{cadre07c}{RGB}{ 24,  48,  29}
\definecolor{cadre07d}{RGB}{ 22,  54,  31}
\definecolor{cadre07e}{RGB}{ 23,  58,  34}

\definecolor{cadre08a}{RGB}{ 53,  55,  25}
\definecolor{cadre08b}{RGB}{ 71,  71,  29}
\definecolor{cadre08c}{RGB}{ 90,  95,  34}
\definecolor{cadre08d}{RGB}{ 99, 110,  36}

\definecolor{cadre09a}{RGB}{ 82,  56,  24}
\definecolor{cadre09b}{RGB}{101,  67,  26}
\definecolor{cadre09c}{RGB}{122,  93,  31}
\definecolor{cadre09d}{RGB}{140, 107,  33}

\definecolor{cadre10a}{RGB}{ 86,  40,  18}
\definecolor{cadre10b}{RGB}{100,  51,  21}
\definecolor{cadre10c}{RGB}{126,  64,  24}
\definecolor{cadre10d}{RGB}{146,  78,  25}
\definecolor{cadre10e}{RGB}{177,  88,  29}
\definecolor{cadre10f}{RGB}{193,  94,  28}

\pgfdeclarehorizontalshading{secteur01}{1.1cm}{
  color(0.0cm)=(rayon01); 
  color(2.5cm)=(rayon02)
}
\pgfdeclarehorizontalshading{secteur02}{1.1cm}{
  color(0.0cm)=(rayon02); 
  color(2.2cm)=(rayon03)
}
\pgfdeclarehorizontalshading{secteur03}{1.7cm}{
  color(0.0cm)=(rayon03); 
  color(1.8cm)=(rayon04)
}
\pgfdeclarehorizontalshading{secteur04}{1.1cm}{
  color(0.0cm)=(rayon04); 
  color(2.2cm)=(rayon05)
}
\pgfdeclarehorizontalshading{secteur05}{1.1cm}{
  color(0.0cm)=(rayon05); 
  color(2.5cm)=(rayon06)
}
\pgfdeclarehorizontalshading{secteur06}{1.1cm}{
  color(0.0cm)=(rayon06); 
  color(2.5cm)=(rayon07)
}
\pgfdeclarehorizontalshading{secteur07}{1.1cm}{
  color(0.0cm)=(rayon07); 
  color(2.2cm)=(rayon08)
}
\pgfdeclarehorizontalshading{secteur08}{1.7cm}{
  color(0.0cm)=(rayon08); 
  color(1.8cm)=(rayon09)
}
\pgfdeclarehorizontalshading{secteur09}{1.1cm}{
  color(0.0cm)=(rayon09); 
  color(2.2cm)=(rayon10)
}
\pgfdeclarehorizontalshading{secteur10}{1.1cm}{
  color(0.0cm)=(rayon10); 
  color(2.5cm)=(rayon01)
}

\newcommand{\systemeLab}{%
  \secteur{105.1}{ 74.9}{secteur03}{ 180}

  \rangee{74.9}{56}{5}
  \cadre{74.9}{56}{1}{white}
  \cadre{74.9}{56}{2}{cadre03b}
  \cadre{74.9}{56}{3}{cadre03c}
  \cadre{74.9}{56}{4}{cadre03d}
  \cadre{74.9}{56}{5}{cadre03e}

  \secteur{74.9}{40.7}{secteur02}{-213}

  \rangee{105.1}{124}{6}
  \cadre{105.1}{124}{1}{white}
  \cadre{105.1}{124}{2}{cadre04b}
  \cadre{105.1}{124}{3}{cadre04c}
  \cadre{105.1}{124}{4}{cadre04d}
  \cadre{105.1}{124}{5}{cadre04e}
  \cadre{105.1}{124}{6}{cadre04f}

  \secteur{139.3}{105.1}{secteur04}{213}

  \rangee{40.7}{19}{6}
  \cadre{40.7}{19}{1}{cadre02a}
  \cadre{40.7}{19}{2}{cadre02b}
  \cadre{40.7}{19}{3}{cadre02c}
  \cadre{40.7}{19}{4}{cadre02d}
  \cadre{40.7}{19}{5}{cadre02e}
  \cadre{40.7}{19}{6}{cadre02f}

  \secteur{40.7}{0.0}{secteur01}{-235}

  \rangee{139.3}{161}{4}
  \cadre{139.3}{161}{1}{cadre05a}
  \cadre{139.3}{161}{2}{cadre05b}
  \cadre{139.3}{161}{3}{cadre05c}
  \cadre{139.3}{161}{4}{cadre05d}

  \secteur{180.0}{139.3}{secteur05}{235}

  \rangee{0}{0}{7}
  \cadre{0}{0}{1}{cadre01a}
  \cadre{0}{0}{2}{cadre01b}
  \cadre{0}{0}{3}{cadre01c}
  \cadre{0}{0}{4}{cadre01d}
  \cadre{0}{0}{5}{cadre01e}
  \cadre{0}{0}{6}{cadre01f}
  \cadre{0}{0}{7}{cadre01g}

  \rangee{180.0}{180}{4}
  \cadre{180.0}{180}{1}{cadre06a}
  \cadre{180.0}{180}{2}{cadre06b}
  \cadre{180.0}{180}{3}{cadre06c}
  \cadre{180.0}{180}{4}{cadre06d}

  % Colonne de cylindres
  \begin{scope}
    \foreach \z in {-4, ..., 4}{
      \cylindre{\z}
    }
  \end{scope}

  \secteur{-180.0}{-139.3}{secteur06}{-55}

  \rangee{-139.3}{-161}{5}
  \cadre{-139.3}{-161}{1}{cadre07a}
  \cadre{-139.3}{-161}{2}{cadre07b}
  \cadre{-139.3}{-161}{3}{cadre07c}
  \cadre{-139.3}{-161}{4}{cadre07d}
  \cadre{-139.3}{-161}{5}{cadre07e}

  \secteur{-139.3}{-105.1}{secteur07}{-33}
  \secteur{-40.7}{0.0}{secteur10}{55}

  \rangee{-40.7}{-19}{6}
  \cadre{-40.7}{-19}{1}{cadre10a}
  \cadre{-40.7}{-19}{2}{cadre10b}
  \cadre{-40.7}{-19}{3}{cadre10c}
  \cadre{-40.7}{-19}{4}{cadre10d}
  \cadre{-40.7}{-19}{5}{cadre10e}
  \cadre{-40.7}{-19}{6}{cadre10f}

  \rangee{-105.1}{-124}{4}
  \cadre{-105.1}{-124}{1}{cadre08a}
  \cadre{-105.1}{-124}{2}{cadre08b}
  \cadre{-105.1}{-124}{3}{cadre08c}
  \cadre{-105.1}{-124}{4}{cadre08d}

  \secteur{-74.9}{-40.7}{secteur09}{33}

  \rangee{-74.9}{-56}{4}
  \cadre{-74.9}{-56}{1}{cadre09a}
  \cadre{-74.9}{-56}{2}{cadre09b}
  \cadre{-74.9}{-56}{3}{cadre09c}
  \cadre{-74.9}{-56}{4}{cadre09d}

  \secteur{-105.1}{-74.9}{secteur08}{0}

  \node at (-2.0,  9.2) {\CIELb} ;
  \node at (10.7,  3.7) {\CIEab} ;
  \node at ( 3.0, -4.6) {\CIEbb} ;
}



\newcommand{\systemeYxy}{%
  \begin{axis}[%
    width=8cm, height=9cm,
    xmin=0, xmax=0.8,
    ymin=0, ymax=0.9,
    axis lines*=left,
    tick align=outside,
    tick style={black, thin},
    ticklabel style={font=\small},
    minor tick num=3,
    xtick={0.0, 0.1, ..., 1.0},
    ytick={0.0, 0.1, ..., 1.0},
    minor tick={0.0, 0.05, ..., 1.0},
    xlabel=\CIExb, ylabel=\CIEyb,
    xlabel near ticks, ylabel near ticks,
    every axis x label/.style={
        at={(ticklabel* cs:1.0)},
        anchor=west,
    },
    every axis y label/.style={
        at={(ticklabel* cs:1.0)},
        anchor=south,
    },
    clip mode=individual,
  ]

    \begin{scope}
      \clip ( 0.0, 0.0) -- ( 0.0, 0.9) -- ( 0.1, 0.9) --
            ( 0.8, 0.2) -- ( 0.8, 0.0) -- cycle ;
      \draw [help lines, step=0.05] 
            ( 0.0, 0.0) grid ( 0.8, 0.9) ;
    \end{scope}
    \draw [help lines] 
          ( 0.0, 0.9) -- ( 0.1, 0.9) -- 
          ( 0.8, 0.2) -- ( 0.8, 0.0) ;
    \draw ( 0.0, 0.9) -- ( 0.0, 0.0) -- ( 0.8, 0.0) ;

    \begin{scope}
      \addplot graphics [
        xmin=0.0036363842, xmax=0.7346900904, 
        ymin=0.0047740307, ymax=0.8340903145
      ]
        {pav} ;
    \end{scope}

    \begin{scope}[thick, line cap=round, line join=round]
      \addplot [thick, mark=none]
         table [x=x, y=y] 
               {\datadir/chro/CIE_espaces2D.dat} ;
      % \draw plot file {CIE_espaces2D.dat} -- cycle ;
    \end{scope}

    \begin{scope}[thin, font=\tiny]
      \draw ( 0.1644117564, 0.0108575583) -- ++ ( 137.64+90:4pt) 
            node [shift={( 137.64+90:5pt)}] {440} ;
      \draw ( 0.1439603960, 0.0297029703) -- ++ ( 133.09+90:4pt) 
            node [shift={( 133.09+90:5pt)}] {460} ;
      \draw ( 0.0912935070, 0.1327020425) -- ++ ( 110.18+90:4pt) 
            node [shift={( 110.18+90:5pt)}] {480} ;
      \draw ( 0.0081680280, 0.5384230705) -- ++ (  94.82+90:4pt) 
            node [shift={(  94.82+90:5pt)}] {500} ;
      \draw ( 0.0743024248, 0.8338030913) -- ++ (   6.75+90:4pt) 
            node [shift={(   6.75+90:5pt)}] {520} ;
      \draw ( 0.2296196726, 0.7543290899) -- ++ ( -38.31+90:4pt) 
            node [shift={( -38.31+90:5pt)}] {540} ;
      \draw ( 0.3731015439, 0.6244508598) -- ++ ( -44.17+90:4pt) 
            node [shift={( -44.17+90:5pt)}] {560} ;
      \draw ( 0.5124863668, 0.4865907881) -- ++ ( -44.90+90:4pt) 
            node [shift={( -44.90+90:5pt)}] {580} ;
      \draw ( 0.6270365998, 0.3724911452) -- ++ ( -44.87+90:4pt) 
            node [shift={( -44.87+90:5pt)}] {600} ;
      \draw ( 0.6915039730, 0.3083422606) -- ++ ( -44.89+90:4pt) 
            node [shift={( -44.89+90:5pt)}] {620} ;
      \draw ( 0.7190329416, 0.2809349515) -- ++ ( -44.95+90:4pt) 
            node [shift={( -44.95+90:5pt)}] {640} ;

      \draw ( 0.1611045796, 0.0137933588) -- ++ ( 138.71+90:2pt) ;  % 445
      \draw ( 0.1566409326, 0.0177048050) -- ++ ( 138.88+90:2pt) ;  % 450
      \draw ( 0.1509854084, 0.0227401933) -- ++ ( 137.24+90:2pt) ;  % 455
      \draw ( 0.1355026712, 0.0398791215) -- ++ ( 126.08+90:2pt) ;  % 465
      \draw ( 0.1241184767, 0.0578025134) -- ++ ( 119.55+90:2pt) ;  % 470
      \draw ( 0.1095943236, 0.0868425112) -- ++ ( 114.06+90:2pt) ;  % 475
      \draw ( 0.0687059213, 0.2007232177) -- ++ ( 106.23+90:2pt) ;  % 485
      \draw ( 0.0453907347, 0.2949759646) -- ++ ( 102.18+90:2pt) ;  % 490
      \draw ( 0.0234599425, 0.4127034791) -- ++ (  98.88+90:2pt) ;  % 495
      \draw ( 0.0038585209, 0.6548231511) -- ++ (  88.65+90:2pt) ;  % 505
      \draw ( 0.0138702461, 0.7501864280) -- ++ (  77.46+90:2pt) ;  % 510
      \draw ( 0.0388518024, 0.8120160214) -- ++ (  52.78+90:2pt) ;  % 515
      \draw ( 0.1141607196, 0.8262069598) -- ++ ( -21.43+90:2pt) ;  % 525
      \draw ( 0.1547220612, 0.8058635454) -- ++ ( -30.15+90:2pt) ;  % 530
      \draw ( 0.1928760979, 0.7816292164) -- ++ ( -34.70+90:2pt) ;  % 535
      \draw ( 0.2657750850, 0.7243239249) -- ++ ( -40.87+90:2pt) ;  % 545
      \draw ( 0.3016037994, 0.6923077624) -- ++ ( -42.54+90:2pt) ;  % 550
      \draw ( 0.3373633329, 0.6588482901) -- ++ ( -43.57+90:2pt) ;  % 555
      \draw ( 0.4087362557, 0.5896068689) -- ++ ( -44.52+90:2pt) ;  % 565
      \draw ( 0.4440624636, 0.5547139028) -- ++ ( -44.76+90:2pt) ;  % 570
      \draw ( 0.4787747912, 0.5202023072) -- ++ ( -44.89+90:2pt) ;  % 575
      \draw ( 0.5447865056, 0.4544341146) -- ++ ( -44.85+90:2pt) ;  % 585
      \draw ( 0.5751513114, 0.4242322349) -- ++ ( -44.90+90:2pt) ;  % 590
      \draw ( 0.6029327856, 0.3964966336) -- ++ ( -44.93+90:2pt) ;  % 595
      \draw ( 0.6482331060, 0.3513949163) -- ++ ( -44.81+90:2pt) ;  % 605
      \draw ( 0.6657635762, 0.3340106512) -- ++ ( -44.80+90:2pt) ;  % 610
      \draw ( 0.6800788497, 0.3197472171) -- ++ ( -44.95+90:2pt) ;  % 615
      \draw ( 0.7006060606, 0.2993006993) -- ++ ( -44.80+90:2pt) ;  % 625
      \draw ( 0.7079177916, 0.2920271089) -- ++ ( -44.89+90:2pt) ;  % 630
      \draw ( 0.7140315971, 0.2859288735) -- ++ ( -44.95+90:2pt) ;  % 635
      \draw ( 0.7230316026, 0.2769483577) -- ++ ( -44.86+90:2pt) ;  % 645
      \draw ( 0.7259923175, 0.2740076825) -- ++ ( -44.91+90:2pt) ;  % 650
      \draw ( 0.7282717283, 0.2717282717) -- ++ ( -45.00+90:2pt) ;  % 655
      \draw ( 0.7299690128, 0.2700309872) -- ++ ( -45.00+90:4pt) ;  % 660
      \draw ( 0.7310893956, 0.2689106044) -- ++ ( -45.00+90:2pt) ;  % 665
      \draw ( 0.7319932998, 0.2680067002) -- ++ ( -45.00+90:2pt) ;  % 670
      \draw ( 0.7327188940, 0.2672811060) -- ++ ( -45.00+90:2pt) ;  % 675
      \draw ( 0.7334169672, 0.2665830328) -- ++ ( -45.00+90:4pt) ;  % 680
      \draw ( 0.7340473003, 0.2659526997) -- ++ ( -45.00+90:2pt) ;  % 685
      \draw ( 0.7343901650, 0.2656098350) -- ++ ( -45.00+90:2pt) ;  % 690
      \draw ( 0.7345916616, 0.2654083384) -- ++ ( -45.00+90:2pt) ;  % 695
    \end{scope}

    \draw [->, >=stealth'] 
          ( 0.58, 0.52) -- ( 0.66, 0.44) 
          node [midway, above right] {\small\WLdb} ;
  \end{axis}
}

