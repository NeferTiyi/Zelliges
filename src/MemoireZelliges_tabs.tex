% .. tabular for chemical composition ..
% ======================================
\newenvironment{cartotab}
  {%
    \smaller%
    \noindent\ignorespaces
    \tabular{R{0.9cm} @{\ =\ } S @{\ $\pm$\ } S @{\hspace*{3em}}
             R{0.9cm} @{\ =\ } S @{\ $\pm$\ } S @{\hspace*{3em}} 
             R{0.9cm} @{\ =\ } S @{\ $\pm$\ } S @{\hspace*{3em}} 
             R{0.9cm} @{\ =\ } S @{\ $\pm$\ } S }%
      \toprule
  }
  {%
      \midrule
        \cartolgntt
      \bottomrule
    \endtabular%
  }
\newcommand{\cartolgn}[3]{\ce{#1}  & #2  & #3}
\newcommand{\cartolgnnd}[1]{\ce{#1}  & \multicolumn{2}{c}{\emph{nd}}}
\newcommand{\cartolgntt}{%
  \multicolumn{10}{r}{\RaggedLeft Total :} & 
  \multicolumn{2}{S}{100.00}
  \tabularnewline
}


% http://tex.stackexchange.com/questions/6910/bottomrule-not-working-in-a-self-made-environment

% .. tabular for chromametry result ..
% ====================================
% \newenvironment{saotab}[1][Zone étudiée]
%   {%
%     \renewcommand{\tabularxcolumn}[1]{M{##1}}
%     \noindent\ignorespaces%
%     \smaller%
%     \setlength{\tabcolsep}{0pt}%
%     \tabularx{1.0\textwidth}
%              {>{\bfseries}M{2cm} @{\hspace*{2em}} 
%               SS @{\hspace*{2em}} 
%               SSS @{\hspace*{2em}} 
%               SSS @{\hspace*{2em}} 
%               >{\bfseries}X}
%         \multirow{2}{2cm}{#1}     & 
%         {$\lambda_D$} & {$P_e$}   &
%         \multirow{2}{2em}{{$Y$}}  & 
%         \multirow{2}{2em}{{$x$}}  & 
%         \multirow{2}{2em}{{$y$}}  & 
%         \multirow{2}{2em}{{$L*$}} & 
%         \multirow{2}{2em}{{$a*$}} & 
%         \multirow{2}{2em}{{$b*$}} & 
%         \multirow{2}{2cm}{Couleur associée}
%       \tabularnewline
%          & 
%         {(\si{\nm})} & {(\si{\percent})} &
%          & & & & & &
%       \tabularnewline
%       \otoprule
%   }
%   {%
%       \bottomrule
%     \endtabularx%
%   }
\newcommand{\chrolgna}[9]{%
  #1 & #2 & #3 & #4 & #5 & #6 & #7 & #8 & #9
}
\newcommand{\chrolgnb}[5]{%
  \mbox{#1} {\normalfont(\mbox{#2} \mbox{\SIrange{#3}{#4}{\nm}#5})}
}

\NewEnviron{chrotab}[1][Zone\\étudiée]{%
  \renewcommand{\tabularxcolumn}[1]{M{##1}}
  \noindent\ignorespaces%
  \smaller%
  \setlength{\tabcolsep}{0pt}%
  \begin{tabularx}{1.0\textwidth}
                  {>{\bfseries}M{2cm} @{\hspace*{2em}}
                   SS @{\hspace*{2em}} 
                   SSS @{\hspace*{2em}} 
                   SSS @{\hspace*{2em}} 
                   >{\bfseries}X}
      \multirow{2}{2cm}{#1}     & 
      {$\lambda_D$} & {$P_e$}   &
      \multirow{2}{2em}{{$Y$}}  & 
      \multirow{2}{2em}{{$x$}}  & 
      \multirow{2}{2em}{{$y$}}  & 
      \multirow{2}{2em}{{$L*$}} & 
      \multirow{2}{2em}{{$a*$}} & 
      \multirow{2}{2em}{{$b*$}} & 
      \multirow{2}{2cm}{Couleur associée}
    \tabularnewline
       & 
      {(\si{\nm})} & {(\si{\percent})} &
       & & & & & &
    \tabularnewline
    \otoprule

    \BODY

    \bottomrule
  \end{tabularx}%
}


\NewEnviron{testtab}{%
  \renewcommand{\tabularxcolumn}[1]{M{##1}}
  \begin{tabularx}{1.0\textwidth}
                  {>{\bfseries}X @{\qquad}
                   SS @{\qquad} 
                   SSS @{\qquad} 
                   SSS @{\qquad} 
                   >{\bfseries}X}
    \toprule

    Zone étudiée & 
    {$\lambda_D$} & {$P_e$} &
    {$Y$}  & {$x$}  & {$y$}  &
    {$L*$} & {$a*$} & {$b*$} &
    Couleur
    \tabularnewline

    \BODY

    \bottomrule
  \end{tabularx}
}
