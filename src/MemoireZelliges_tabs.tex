%!TEX root = MemoireZelliges.tex

% .. tabular for chemical composition ..
% ======================================
\newenvironment{cartotab}
  {%
    \smaller%
    \noindent\ignorespaces
    \tabular{R{0.9cm} @{\ =\ } S @{\ $\pm$\ } S @{\hspace*{3em}}
             R{0.9cm} @{\ =\ } S @{\ $\pm$\ } S @{\hspace*{3em}} 
             R{0.9cm} @{\ =\ } S @{\ $\pm$\ } S @{\hspace*{3em}} 
             R{0.9cm} @{\ =\ } S @{\ $\pm$\ } S }%
      \toprule
  }
  {%
      \midrule
        \cartolgntt
      \bottomrule
    \endtabular%
  }
\newcommand{\cartolgn}[3]{\ce{#1}  & #2  & #3}
\newcommand{\cartolgnnd}[1]{\ce{#1}  & \multicolumn{2}{c}{\emph{nd}}}
\newcommand{\cartolgntt}{%
  \multicolumn{10}{r}{\RaggedLeft Total :} & 
  \multicolumn{2}{S}{100.00}
  \tabularnewline
}


% .. tabular for chromametry result ..
% ====================================
\newcommand{\chrolgna}[9]{%
  #1 & #2 & #3 & #4 & #5 & #6 & #7 & #8 & #9
}
\newcommand{\chrolgnb}[5]{%
  \mbox{#1} {\normalfont(\mbox{#2} \mbox{\SIrange{#3}{#4}{\nm}#5})}
}
\NewEnviron{chrotab}[1][Zone\\étudiée]{%
  \renewcommand{\tabularxcolumn}[1]{M{##1}}
  % \sisetup{%
  %   table-figures-integer=4,
  %   table-figures-decimal=3
  % }
  \noindent\ignorespaces%
  \smaller%
  \setlength{\tabcolsep}{0pt}%
  \begin{tabularx}{1.0\textwidth}
                  {>{\bfseries}M{2cm}                @{\qquad}
                   S@{\hspace{1ex}}S                 @{\qquad} 
                   S@{\hspace{1ex}}S@{\hspace{1ex}}S @{\qquad} 
                   S@{\hspace{1ex}}S@{\hspace{1ex}}S @{\qquad} 
                   >{\bfseries}X}
      \multirow{2}{2cm}{#1}     & 
      {$\lambda_D$} & {\Pe}   &
      \multirow{2}{2em}{{\CIEY}}  & 
      \multirow{2}{2em}{{\CIEx}}  & 
      \multirow{2}{2em}{{\CIEy}}  & 
      \multirow{2}{2em}{{\CIEL}} & 
      \multirow{2}{2em}{{\CIEa}} & 
      \multirow{2}{2em}{{\CIEb}} & 
      \multirow{2}{2cm}{Couleur associée}
    \tabularnewline
       & 
      {(\si{\nm})} & {(\si{\percent})} &
       & & & & & &
    \tabularnewline
    \otoprule

    \BODY

    \bottomrule
  \end{tabularx}%
}
