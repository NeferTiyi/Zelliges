%!TEX root = MemoireZelliges.tex

{
\large
\noindent\SL

\noindent DESS Méthodes Physiques Appliquées à l'Archéologie et à la 
Muséographie -- Juin 2000

\vfill

\begin{center}
\bfseries\Large
Programme PACT ARCHI-MED Glaçures -- II

\vspace*{-0.75\baselineskip}

\rule{.5\textwidth}{1pt}

\bigskip

\MakeUppercase{Contribution à la réhabilitation architecturale du 
\PaM, Meknès, Maroc, \siecle{xvii}}

\bigskip

Recherche des caractéristiques physiques de zelliges de pavement
\end{center}

\vfill

Le travail présenté ici s'intègre dans le cadre du programme 
\emph{PACT ARCHI-MED Glaçures} qui a pour but l'étude de la céramique 
glaçurée architecturale du monde méditerranéen. Cette étude porte sur 
cinq échantillons de zelliges (bleu, vert, miel, noir et blanc) 
provenant du \PaM de Meknès, \siecle{xvii}.

L'étude de la texture en section, en lumière naturelle et en \CL, 
a permis de mettre en évidence deux groupes d'échantillons se 
distinguant par la couleur de leur support céramique, la granulométrie 
des inclusions et la présence ou l'absence de liseré luminescent à 
l'interface glaçure/terre cuite : d'une part les zelliges vert, miel, 
noir et blanc, d'autre part, le zellige bleu, de plus grandes 
dimensions que les autres, avec un support céramique plus rouge, 
contenant des inclusions plus grossières. L'imagerie en \MEB[ie] donne 
accès à la microtexture des échantillons et permet de corréler un 
liseré qui luminesce en jaune en \CL à la présence de cristaux de 
dévitrification dans la zone d'interface terre cuite/glaçure.

L'analyse quantitative en \EDS montre la grande homogénéité des matériaux, tant des glaçures plombifères, que des terres cuites de type calcique, et permet de détecter la présence d'étain, utilisé comme opacifiant dans deux des cinq glaçures (blanche et bleue).

La présence de phases cristallines de haute température dans les terres cuites, détectées en \DX sur poudre peut donner des indications quant aux températures de cuisson atteintes, entre \SIrange[range-phrase=\ et\ ]{850}{950}{\degC}.

La \SAO nous a permis de déterminer les éléments chromogènes utilisés pour colorer les glaçures (bleue : \ce{Co^2+} ; verte : \ce{Cu^2+} dans une glaçure plombifère ; miel : \ce{Fe^3+} ; noire : \ce{Mn^3+}) et la \CHRO donne une définition objective de ces couleurs.

L'ensemble de ces données nous a permis de proposer l'hypothèse de deux techniques de fabrication différentes, selon la destination du zellige, à partir d'une même matière première : une technique assez simple pour les zelliges de grande taille cernant le décor, une technique plus élaborée pour les zelliges moins épais formant le décor à proprement parler.

\vfill

\begin{motsclef}
   zellige, céramique glaçurée, architecture, Maroc, Meknès, 
   \PaM, \siecle{xvii}, patrimoine culturel, méthodes physiques 
   de caractérisation
\end{motsclef}
}
